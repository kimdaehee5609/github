
%	-------------------------------------------------------------------------------
%
%			작성		
%				2020년 
%				12월 
%				13일 
%				일
%
%
%
%
%
%
%	-------------------------------------------------------------------------------

%	\documentclass[10pt,xcolor=pdftex,dvipsnames,table]{beamer}
%	\documentclass[10pt,blue,xcolor=pdftex,dvipsnames,table,handout]{beamer}
%	\documentclass[14pt,blue,xcolor=pdftex,dvipsnames,table,handout]{beamer}
	\documentclass[aspectratio=1610,20pt,xcolor=pdftex,dvipsnames,table,handout]{beamer}
%	\documentclass[aspectratio=169,17pt,xcolor=pdftex,dvipsnames,table,handout]{beamer}
%	\documentclass[aspectratio=149,17pt,xcolor=pdftex,dvipsnames,table,handout]{beamer}
%	\documentclass[aspectratio=54,17pt,xcolor=pdftex,dvipsnames,table,handout]{beamer}
%	\documentclass[aspectratio=43,17pt,xcolor=pdftex,dvipsnames,table,handout]{beamer}
%	\documentclass[aspectratio=32,17pt,xcolor=pdftex,dvipsnames,table,handout]{beamer}

		% Font Size
		%	default font size : 11 pt
		%	8,9,10,11,12,14,17,20
		%
		% 	put frame titles 
		% 		1) 	slideatop
		%		2) 	slide centered
		%
		%	navigation bar
		% 		1)	compress
		%		2)	uncompressed
		%
		%	Color
		%		1) blue
		%		2) red
		%		3) brown
		%		4) black and white	
		%
		%	Output
		%		1)  	[default]	
		%		2)	[handout]		for PDF handouts
		%		3) 	[trans]		for PDF transparency
		%		4)	[notes=hide/show/only]

		%	Text and Math Font
		% 		1)	[sans]
		% 		2)	[sefif]
		%		3) 	[mathsans]
		%		4)	[mathserif]


		%	---------------------------------------------------------	
		%	슬라이드 크기 설정 ( 128mm X 96mm )
		%	---------------------------------------------------------	
%			\setbeamersize{text margin left=2mm}
%			\setbeamersize{text margin right=2mm}

		%	---------------------------------------------------------	
		%	슬라이드 크기 설정 ( 128mm X 96mm )
		%	---------------------------------------------------------	

%			% Format presentation size to A4
%			\usepackage[size=a4]{beamerposter}		% A4용지 크기 사용
			\geometry{paper=a5paper}
%			% Format presentation size to A4 길게
%			\geometry{paper=a4paper, landscape}

			\setbeamersize{text margin left=10mm}
			\setbeamersize{text margin right=10mm}


	%	========================================================== 	Package
		\usepackage{kotex}						% 한글 사용
		\usepackage{amssymb,amsfonts,amsmath}	% 수학 수식 사용
		\usepackage{color}					%
		\usepackage{colortbl}					%

		\usepackage{caption}		% 테이블 사용 건 2020.10.22
		\usepackage{tabularx} 	% 테이블 사용 건 2020.10.22


	%		========================================================= 	note 옵션인 
	%			\setbeameroption{show only notes}
		

	%		========================================================= 	Theme

		%	---------------------------------------------------------	
		%	전체 테마
		%	---------------------------------------------------------	
		%	테마 명명의 관례 : 도시 이름
%			\usetheme{default}			%
%			\usetheme{Madrid}    		%
%			\usetheme{CambridgeUS}    	% -red, no navigation bar
%			\usetheme{Antibes}			% -blueish, tree-like navigation bar

		%	----------------- table of contents in sidebar
			\usetheme{Berkeley}		% -blueish, table of contents in sidebar
									% 개인적으로 마음에 듬

%			\usetheme{Marburg}			% - sidebar on the right
%			\usetheme{Hannover}		% 왼쪽에 마크
%			\usetheme{Berlin}			% - navigation bar in the headline
%			\usetheme{Szeged}			% - navigation bar in the headline, horizontal lines
%			\usetheme{Malmoe}			% - section/subsection in the headline

%			\usetheme{Singapore}
%			\usetheme{Amsterdam}

		%	---------------------------------------------------------	
		%	색 테마
		%	---------------------------------------------------------	
%			\usecolortheme{albatross}	% 바탕 파란
%			\usecolortheme{crane}		% 바탕 흰색
%			\usecolortheme{beetle}		% 바탕 회색
%			\usecolortheme{dove}		% 전체적으로 흰색
%			\usecolortheme{fly}		% 전체적으로 회색
%			\usecolortheme{seagull}	% 휜색
%			\usecolortheme{wolverine}	& 제목이 노란색
%			\usecolortheme{beaver}

		%	---------------------------------------------------------	
		%	Inner Color Theme 			내부 색 테마 ( 블록의 색 )
		%	---------------------------------------------------------	

%			\usecolortheme{rose}		% 흰색
%			\usecolortheme{lily}		% 색 안 칠한다
%			\usecolortheme{orchid} 	% 진하게

		%	---------------------------------------------------------	
		%	Outter Color Theme 		외부 색 테마 ( 머리말, 고리말, 사이드바 )
		%	---------------------------------------------------------	

%			\usecolortheme{whale}		% 진하다
%			\usecolortheme{dolphin}	% 중간
%			\usecolortheme{seahorse}	% 연하다

		%	---------------------------------------------------------	
		%	Font Theme 				폰트 테마
		%	---------------------------------------------------------	
%			\usfonttheme{default}		
			\usefonttheme{serif}			
%			\usefonttheme{structurebold}			
%			\usefonttheme{structureitalicserif}			
%			\usefonttheme{structuresmallcapsserif}			



		%	---------------------------------------------------------	
		%	Inner Theme 				
		%	---------------------------------------------------------	

%			\useinnertheme{default}
			\useinnertheme{circles}		% 원문자			
%			\useinnertheme{rectangles}		% 사각문자			
%			\useinnertheme{rounded}			% 깨어짐
%			\useinnertheme{inmargin}			




		%	---------------------------------------------------------	
		%	이동 단추 삭제
		%	---------------------------------------------------------	
%			\setbeamertemplate{navigation symbols}{}

		%	---------------------------------------------------------	
		%	문서 정보 표시 꼬리말 적용
		%	---------------------------------------------------------	
%			\useoutertheme{infolines}


			
	%	---------------------------------------------------------- 	배경이미지 지정
%			\pgfdeclareimage[width=\paperwidth,height=\paperheight]{bgimage}{./fig/Chrysanthemum.jpg}
%			\setbeamertemplate{background canvas}{\pgfuseimage{bgimage}}

		%	---------------------------------------------------------	
		% 	본문 글꼴색 지정
		%	---------------------------------------------------------	
%			\setbeamercolor{normal text}{fg=purple}
%			\setbeamercolor{normal text}{fg=red!80}	% 숫자는 투명도 표시


		%	---------------------------------------------------------	
		%	itemize 모양 설정
		%	---------------------------------------------------------	
%			\setbeamertemplate{items}[ball]
%			\setbeamertemplate{items}[circle]
%			\setbeamertemplate{items}[rectangle]






		\setbeamercovered{dynamic}





		% --------------------------------- 	문서 기본 사항 설정
		\setcounter{secnumdepth}{3} 		% 문단 번호 깊이
		\setcounter{tocdepth}{3} 			% 문단 번호 깊이




% ------------------------------------------------------------------------------
% Begin document (Content goes below)
%
%
% ------------------------------------------------------------------------------
	\begin{document}
	

			\title{ GitHub }
			\author{ 김대희 }
			\date{ 
					2020년 
					12월 
					21일
					월요일  
					} 


% -----------------------------------------------------------------------------
%		개정 내용
% -----------------------------------------------------------------------------
%
%		2020년 12월 13일 첫제작
%		2020년 12월 14일 프라스틱 링 바인더용으로 편집
%		2020.12.21 전반적인 내용 수정


	%	==========================================================
	%
	%	----------------------------------------------------------
		\begin{frame}[plain]
		\titlepage
		\end{frame}


		\begin{frame} [plain]{목차}
		\tableofcontents%

%			\setlength{\leftmargini}{ 2em}			
%			\begin{itemize}
%
%				\item [part1] \ref{part1}	개요
%				\item [part2] \ref{part2}	연락처
%				\item [part3] \ref{part3}	리눅스
%				\item [part4] \ref{part4}	좋은 문구
%%				\item [part5] \ref{part5}	출입문			
%%				\item [part6] \ref{part6}	일반사이트		
%%				\item [part7] \ref{part7}	쇼핑 사이트		
%
%			\end{itemize}
		\end{frame}


	%	----------------------------------------------------------
		\begin{frame} [t,plain]
		\frametitle{}
		\end{frame}			

	%	----------------------------------------------------------
		\begin{frame} [t,plain]
		\frametitle{}
		\end{frame}			
					



	%	========================================================== 1 개요
		\part{개요 }
		\frame{\partpage}

\label{part1} 	%  개요


	%	---------------------------------------------------------- 2 목차
		\begin{frame} [plain]{목차}
		\tableofcontents%
		\end{frame}


		

	%	----------------------------------------------------------  개요
	%		Frame
	%	----------------------------------------------------------
		\section{개요}


	%	---------------------------------------------------------- 3 개요
		\begin{frame} [t,plain]
%		\frametitle{개요}

			\begin{block} {개요}
			\setlength{\leftmargini}{2em}			
			\begin{itemize}
				\item 2008
				\item Octo Cat
				\item 리누스 토발즈가 리눅스의 소스코드 관리응 위해
				\item Linus Torralds
				\item 깃은 지옥에서 온 관리자
			\end{itemize}
			\end{block}						

		\end{frame}						


	%	---------------------------------------------------------- 
	%		Frame
	%	----------------------------------------------------------
		\section{목적}

	%	---------------------------------------------------------- 4 	목적
		\begin{frame} [t,plain]
%		\frametitle{목적}

			\begin{block} {목적}
			\setlength{\leftmargini}{2em}			
			\begin{itemize}
				\item 버전관리 version control
				\item 백업 Backup
				\item 협업 callaboration

			\end{itemize}
			\end{block}						

		\end{frame}			


	%	---------------------------------------------------------- 단축키
	%		Frame
	%	----------------------------------------------------------
		\section{단축키}

	%	---------------------------------------------------------- 5	단축키
		\begin{frame} [t,plain]
%		\frametitle{단축키}

			\begin{block} {단축키}
			\setlength{\leftmargini}{2em}			
			\begin{itemize}
				\item 붙이기 shift + Ins
				\item Full screen 	Alt+F11
				\item Flip screen 	Alt+F12 \\.
				\item Copy 	\hrulefill 	Ctrl + Ins
				\item Paste	\hrulefill 	Shift + Ins
				\item Select All \hrulefill 	
				\item Save as Image \hrulefill 	\\.

				\item Search	\hrulefill 	Alt + F3
				\item Reset	\hrulefill 	Alt + F8

				\item Default size	\hrulefill 	Alt + F3
				\item Scrollbar		\hrulefill 	
				\item Full Screen	\hrulefill 	Alt + F11
				\item Flip Screen	\hrulefill 	Alt + F12 \\ .

				\item Option		\hrulefill 	
			\end{itemize}
			\end{block}						

		\end{frame}			


	%	---------------------------------------------------------- 깃  회원 가입
	%		Frame
	%	----------------------------------------------------------
		\section{깃  회원 가입}


	%	---------------------------------------------------------- 6	깃  회원 가입
		\begin{frame} [t,plain]
%		\frametitle{깃  회원 가입}

			\begin{block} {깃  회원 가입}
			\setlength{\leftmargini}{2em}			
			\begin{itemize}
				\item 사용자 이름 \\ kim dae hee 5609
				\item 이메일 주소 \\ h 010 3839 5609 @ g mail . com
				\item 비번 \\  h . 789 456 123
			\end{itemize}
			\end{block}						

		\end{frame}			


	%	---------------------------------------------------------- 깃 환경설정
	%		Frame
	%	----------------------------------------------------------
		\section{깃 환경설정}


	%	---------------------------------------------------------- 7	깃 환경설정하기
		\begin{frame} [t,plain]
%		\frametitle{깃 환경설정}

			\begin{block} {깃 환경설정}
			\setlength{\leftmargini}{2em}			
			\begin{itemize}
				\item git config
				\item git config - - global user.name "kimdaehee5609"
				\item git config - - global user.email "h01038395609@gmail.com"
				\item 
			\end{itemize}
			\end{block}						


		\end{frame}			


	%	---------------------------------------------------------- 깃 사용 환경
	%		Frame
	%	----------------------------------------------------------
		\section{깃 사용 환경}

	%	---------------------------------------------------------- 7	깃 환경설정하기
		\begin{frame} [t,plain]
		\frametitle{깃 사용 설정}

			\begin{block} {깃 GUI : 그래픽 유저 인터페이스}
			\setlength{\leftmargini}{2em}			
			\begin{itemize}
				\item 
				\item 
				\item 
			\end{itemize}
			\end{block}						

			\begin{block} {깃 CLI : 커멘드 라인 인터페이스 }
			\setlength{\leftmargini}{2em}			
			\begin{itemize}
				\item 
				\item 
				\item 
			\end{itemize}
			\end{block}						

		\end{frame}			




	%	========================================================== %	종류
	%												%
	%												%
	%												%
	%	========================================================== %	1
		\part{종류 }
		\frame{\partpage}

\label{part1} 	%  차 시간 정리


	%	---------------------------------------------------------- 2 목차
		\begin{frame} [plain]{목차}
		\tableofcontents%
		\end{frame}


	%	---------------------------------------------------------- 깃 프로그램 종류
	%		Frame
	%	----------------------------------------------------------
		\section{깃 프로그램 종류}


	%	---------------------------------------------------------- 3	깃 프로그램 종류
		\begin{frame} [t,plain]
%		\frametitle{깃 프로그램 종류}

			\begin{block} {깃 프로그램 종류}
			\setlength{\leftmargini}{2em}			
			\begin{itemize}
				\item htts : // git - scm .  com
				\item Git-2.29.2.3-64-bit 
			\end{itemize}
			\end{block}						

		\end{frame}						


	%	---------------------------------------------------------- 깃허브 데스크톱
	%		Frame
	%	----------------------------------------------------------
		\section{깃허브 데스크톱}

	%	---------------------------------------------------------- 4	깃허브 데스크톱
		\begin{frame} [t,plain]
%		\frametitle{깃허브 데스크톱}

			\begin{block} {깃허브 데스크톱}
			\setlength{\leftmargini}{2em}			
			\begin{itemize}
				\item 
			\end{itemize}
			\end{block}						

		\end{frame}						


	%	---------------------------------------------------------- 토터스깃
	%		Frame
	%	----------------------------------------------------------
		\section{토터스깃}


	%	---------------------------------------------------------- 5	토터스깃
		\begin{frame} [t,plain]
%		\frametitle{토터스깃}

			\begin{block} {토터스깃}
			\setlength{\leftmargini}{2em}			
			\begin{itemize}
				\item 
			\end{itemize}
			\end{block}						

		\end{frame}						

	%	---------------------------------------------------------- 소스트리
	%		Frame
	%	----------------------------------------------------------
		\section{소스트리}


	%	---------------------------------------------------------- 6 	소스트리
		\begin{frame} [t,plain]
%		\frametitle{소스트리}

			\begin{block} {소스트리}
			\setlength{\leftmargini}{2em}			
			\begin{itemize}
				\item https : // www.sourcetreeapp.com /
				\item SourceTreeSetup-3.3.9.exe
			\end{itemize}
			\end{block}						

		\end{frame}						


	%	---------------------------------------------------------- 비주얼 스튜디오 코드
	%		Frame
	%	----------------------------------------------------------
		\section{비주얼 스튜디오 코드}


	%	---------------------------------------------------------- 7	비주얼 스튜디오 코드
		\begin{frame} [t,plain]
%		\frametitle{비주얼 스튜디오 코드}

			\begin{block} {비주얼 스튜디오 코드}
			\setlength{\leftmargini}{2em}			
			\begin{itemize}
				\item https://code.visualstudio.com/
				\item VSCodeUserSetup-x64-1.52.1.exe
				\item  한국어 설정

			\end{itemize}
			\end{block}						

		\end{frame}						


	%	---------------------------------------------------------- VIM
	%		Frame
	%	----------------------------------------------------------
		\section{VIM}


	%	---------------------------------------------------------- 8	VIM
		\begin{frame} [t,plain]
%		\frametitle{VIM}

			\begin{block} {VIM}
			\setlength{\leftmargini}{2em}			
			\begin{itemize}
				\item vim
				\item : 
				\item : w
				\item : q
				\item : wq
				\item : q!
			\end{itemize}
			\end{block}						

		\end{frame}						



	%	========================================================== %	리눅스
	%												%
	%												%
	%												%
	%	========================================================== %	1
		\part{리눅스 }
		\frame{\partpage}

\label{part3} 	%  리눅스


	%	---------------------------------------------------------- 2 목차
		\begin{frame} [plain]{목차}
		\tableofcontents%
		\end{frame}
		



	%	---------------------------------------------------------- 리눅스
	%		Frame
	%	----------------------------------------------------------
		\section{리눅스}


	%	---------------------------------------------------------- 3
		\begin{frame} [t,plain]
%		\frametitle{리눅스}

			\begin{block} {리눅스}
			\setlength{\leftmargini}{2em}			
			\begin{itemize}
				\item 
			\end{itemize}
			\end{block}						

		\end{frame}						


	%	---------------------------------------------------------- 4	현재 디렉토리 살펴보기
		\begin{frame} [t,plain]
%		\frametitle{현재 디렉토리 살펴보기}



			\begin{block} {현재 디렉토리 살펴보기}
			\setlength{\leftmargini}{2em}			
			\begin{itemize}
				\item 
			\end{itemize}
			\end{block}						

		\end{frame}			

	%	---------------------------------------------------------- 5	디렉토리 이동하기
		\begin{frame} [t,plain]
%		\frametitle{디렉토리 이동하기}

			\begin{block} {디렉토리 이동하기}
			\setlength{\leftmargini}{2em}			
			\begin{itemize}
				\item 
			\end{itemize}
			\end{block}						

		\end{frame}			



	%	---------------------------------------------------------- 6	디렉토리 만들기 삭제하기
		\begin{frame} [t,plain]
%		\frametitle{디렉토리 만들기 삭제하기}

			\begin{block} {디렉토리 만들기}
			\setlength{\leftmargini}{2em}			
			\begin{itemize}
				\item 
			\end{itemize}
			\end{block}						


			\begin{block} {디렉토리 삭제하기}
			\setlength{\leftmargini}{2em}			
			\begin{itemize}
				\item 
			\end{itemize}
			\end{block}						

		\end{frame}			



%파일 시스템 탐색을 위한 리눅스 명령어
	%	---------------------------------------------------------- 7	파일 시스템 탐색을 위한 리눅스 명령어
		\begin{frame} [t,plain]
%		\frametitle{파일 시스템 탐색을 위한 리눅스 명령어}

			\begin{block} {파일 시스템 탐색을 위한 리눅스 명령어}
			\setlength{\leftmargini}{2em}			
			\begin{itemize}
				\item 
			\end{itemize}
			\end{block}						

		\end{frame}			

%시스템 조작을 위한 리눅스 명령어
	%	---------------------------------------------------------- 8	시스템 조작을 위한 리눅스 명령어
		\begin{frame} [t,plain]
%		\frametitle{시스템 조작을 위한 리눅스 명령어}

			\begin{block} {시스템 조작을 위한 리눅스 명령어}
			\setlength{\leftmargini}{2em}			
			\begin{itemize}
				\item 
			\end{itemize}
			\end{block}						

		\end{frame}			

%파일 관리를 위한 리눅스 명령어
	%	---------------------------------------------------------- 9	파일 관리를 위한 리눅스 명령어
		\begin{frame} [t,plain]
%		\frametitle{파일 관리를 위한 리눅스 명령어}

			\begin{block} {파일 관리를 위한 리눅스 명령어}
			\setlength{\leftmargini}{2em}			
			\begin{itemize}
				\item 
			\end{itemize}
			\end{block}						

		\end{frame}			

%지루할 때 탐색할 수 있는 재미있는 리눅스 명령어
	%	---------------------------------------------------------- 10	지루할 때 탐색할 수 있는 재미있는 리눅스 명령어
		\begin{frame} [t,plain]
%		\frametitle{지루할 때 탐색할 수 있는 재미있는 리눅스 명령어}

			\begin{block} {지루할 때 탐색할 수 있는 재미있는 리눅스 명령어}
			\setlength{\leftmargini}{2em}			
			\begin{itemize}
				\item 
			\end{itemize}
			\end{block}						

		\end{frame}			

%네트워크 관리자에게 가장 많이 사용되는 리눅스 명령어
	%	---------------------------------------------------------- 11	네트워크 관리자에게 가장 많이 사용되는 리눅스 명령어
		\begin{frame} [t,plain]
%		\frametitle{네트워크 관리자에게 가장 많이 사용되는 리눅스 명령어}

			\begin{block} {네트워크 관리자에게 가장 많이 사용되는 리눅스 명령어}
			\setlength{\leftmargini}{2em}			
			\begin{itemize}
				\item 
			\end{itemize}
			\end{block}						

		\end{frame}			

%리눅스 명령어 검색 및 정규 표현식
	%	---------------------------------------------------------- 12	리눅스 명령어 검색 및 정규 표현식
		\begin{frame} [t,plain]
%		\frametitle{리눅스 명령어 검색 및 정규 표현식}

			\begin{block} {리눅스 명령어 검색 및 정규 표현식}
			\setlength{\leftmargini}{2em}			
			\begin{itemize}
				\item 
			\end{itemize}
			\end{block}						

		\end{frame}			

%IO 및 소유권을 다루는 리눅스 명령어
	%	---------------------------------------------------------- 13	IO 및 소유권을 다루는 리눅스 명령어
		\begin{frame} [t,plain]
%		\frametitle{IO 및 소유권을 다루는 리눅스 명령어}

			\begin{block} {IO 및 소유권을 다루는 리눅스 명령어}
			\setlength{\leftmargini}{2em}			
			\begin{itemize}
				\item 
			\end{itemize}
			\end{block}						

		\end{frame}			

%일상적인 사용을 위한 기타 명령어
	%	---------------------------------------------------------- 14	일상적인 사용을 위한 기타 명령어
		\begin{frame} [t,plain]
%		\frametitle{일상적인 사용을 위한 기타 명령어}

			\begin{block} {일상적인 사용을 위한 기타 명령어}
			\setlength{\leftmargini}{2em}			
			\begin{itemize}
				\item 
			\end{itemize}
			\end{block}						

		\end{frame}			


	%	---------------------------------------------------------- 15
		\begin{frame} [t,plain]
		\end{frame}			


	%	---------------------------------------------------------- 16
		\begin{frame} [t,plain]
		\end{frame}			


	%	========================================================== 1	VIM에서 텍스트 문서 만들기
		\part{VIM에서 텍스트 문서 만들기}
		\frame{\partpage}

\label{part1} 	%  VIM에서 텍스트 문서 만들기


	%	---------------------------------------------------------- 2 목차
		\begin{frame} [plain]{목차}
		\tableofcontents%
		\end{frame}
		
	%	---------------------------------------------------------- 3
		\begin{frame} [t,plain]

			\begin{block} {일상적인 사용을 위한 기타 명령어}
			\setlength{\leftmargini}{2em}			
			\begin{itemize}
				\item 
			\end{itemize}
			\end{block}						

		\end{frame}			

	%	---------------------------------------------------------- 4
		\begin{frame} [t,plain]

			\begin{block} {VIM ex 모드 명령}
			\setlength{\leftmargini}{2em}			
			\begin{itemize}
				\item 
			\end{itemize}
			\end{block}						

		\end{frame}			



	%	---------------------------------------------------------- 5
		\begin{frame} [t,plain]

			\begin{block} {텍스트 문서 내용 확인하기}
			\setlength{\leftmargini}{2em}			
			\begin{itemize}
				\item 
			\end{itemize}
			\end{block}						


		\end{frame}			


	%	---------------------------------------------------------- 6
		\begin{frame} [t,plain]

			\begin{block} {}
			\setlength{\leftmargini}{2em}			
			\begin{itemize}
				\item 
			\end{itemize}
			\end{block}						


		\end{frame}			

	%	---------------------------------------------------------- 7
		\begin{frame} [t,plain]

			\begin{block} {}
			\setlength{\leftmargini}{2em}			
			\begin{itemize}
				\item 
			\end{itemize}
			\end{block}						


		\end{frame}			

	%	---------------------------------------------------------- 8
		\begin{frame} [t,plain]

			\begin{block} {깃에서 기본 편집기 변경하기}
			\setlength{\leftmargini}{2em}			
			\begin{itemize}
				\item git config --global core.editor "notepad++"
			\end{itemize}
			\end{block}						

		\end{frame}			




	%	========================================================== 깃 기본용어
	%												%
	%												%
	%												%
	%	========================================================== 1

		\part{깃 기본 용어}
		\frame{\partpage}

\label{part1} 	%  깃 기본 용어


	%	---------------------------------------------------------- 2 목차
		\begin{frame} [plain]{목차}
		\tableofcontents%
		\end{frame}



	%	---------------------------------------------------------- 깃 자주 사용하는 Git 명령어
	%		Frame
	%	----------------------------------------------------------
		\section{깃 자주 사용하는 Git 명령어}

	%	---------------------------------------------------------- page		3
		\begin{frame} [t,plain]
		\frametitle{깃 자주 사용하는 Git 명령어}

			\begin{block} {깃 자주 사용하는 Git 명령어}
			\setlength{\leftmargini}{2em}			
			\begin{itemize}
				\item 커밋 	\hrulefill commit
				\item 풀	 	\hrulefill pull
				\item 푸시 	\hrulefill push
				\item 패치 	\hrulefill Patch
				\item 브랜치	\hrulefill Bramch
				\item 병합		\hrulefill Merge
				\item 스태시	\hrulefill Stach
				\item 패기		\hrulefill 
				\item 태그		\hrulefill Tag
				\item 콜론		\hrulefill Clone
				\item 
			\end{itemize}
			\end{block}						
		\end{frame}			


	%	---------------------------------------------------------- page		4
		\begin{frame} [t,plain]
			\begin{block} {커밋 	\hrulefill commit}
			\setlength{\leftmargini}{2em}			
			\begin{itemize}
				\item 스테이지에 올린 파일들을 한 묶음으로 스냅샷을 찍음
				\item 
				\item 
				\item 
				\item 
				\item 
			\end{itemize}
			\end{block}						
		\end{frame}			

	%	---------------------------------------------------------- page		5
		\begin{frame} [t,plain]
			\begin{block} {풀	 	\hrulefill pull}
			\setlength{\leftmargini}{2em}			
			\begin{itemize}
				\item 원격저장소 Remote Repository에 있는 모든 커밋을 다운로드 받음 
				\item 
				\item 
				\item 
				\item 
				\item 
			\end{itemize}
			\end{block}						
		\end{frame}			

	%	---------------------------------------------------------- page		6
		\begin{frame} [t,plain]
			\begin{block} {푸시 	\hrulefill push}
			\setlength{\leftmargini}{2em}			
			\begin{itemize}
				\item 로컬저장소 에 있는 커밋을 원격저장소에 업로드 함
				\item Local Repository
				\item Remote Repository
				\item 
				\item 
				\item 
			\end{itemize}
			\end{block}						
		\end{frame}			

	%	---------------------------------------------------------- page		7
		\begin{frame} [t,plain]
			\begin{block} {패치 	\hrulefill Patch}
			\setlength{\leftmargini}{2em}			
			\begin{itemize}
				\item 새로 고침
				\item 
				\item 
				\item 
				\item 
				\item 
			\end{itemize}
			\end{block}						
		\end{frame}			

	%	---------------------------------------------------------- page		8
		\begin{frame} [t,plain]
			\begin{block} {브랜치	\hrulefill Bramch}
			\setlength{\leftmargini}{2em}			
			\begin{itemize}
				\item 새로운 브랜치를 생성하거나 삭제함 
				\item 
				\item 
				\item 
				\item 
				\item 
			\end{itemize}
			\end{block}						
		\end{frame}			

	%	---------------------------------------------------------- page		9
		\begin{frame} [t,plain]
			\begin{block} {병합		\hrulefill Merge}
			\setlength{\leftmargini}{2em}			
			\begin{itemize}
				\item 두개의 브랜치를 하나로 합침 
				\item 
				\item 
				\item 
				\item 
				\item 
			\end{itemize}
			\end{block}						
		\end{frame}			

	%	---------------------------------------------------------- page		10
		\begin{frame} [t,plain]
			\begin{block} {스태시	\hrulefill Stach}
			\setlength{\leftmargini}{2em}			
			\begin{itemize}
				\item 작업하던 도중 브랜치를 바꾸거나 할때 Tracked 상태인 파일을 임시 저장함 
				\item 
				\item 
				\item 
				\item 
				\item 
			\end{itemize}
			\end{block}						
		\end{frame}			

	%	---------------------------------------------------------- page		11
		\begin{frame} [t,plain]
			\begin{block} {태그 	\hrulefill Tag}
			\setlength{\leftmargini}{2em}			
			\begin{itemize} 
				\item 보통 배포할때 버전을 태그로 스냅샷을 남김 
				\item 
				\item 
				\item 
				\item 
				\item 
			\end{itemize}
			\end{block}						
		\end{frame}			

	%	---------------------------------------------------------- page		12
		\begin{frame} [t,plain]
			\begin{block} {패기}
			\setlength{\leftmargini}{2em}			
			\begin{itemize}
				\item 
				\item 
				\item 
				\item 
				\item 
				\item 
			\end{itemize}
			\end{block}						
		\end{frame}			

	%	---------------------------------------------------------- page		13
		\begin{frame} [t,plain]
			\begin{block} {태그		\hrulefill Tag}
			\setlength{\leftmargini}{2em}			
			\begin{itemize}
				\item 
				\item 
				\item 
				\item 
				\item 
				\item 
			\end{itemize}
			\end{block}						
		\end{frame}			

	%	---------------------------------------------------------- page		14
		\begin{frame} [t,plain]
			\begin{block} {콜론		\hrulefill Clone}
			\setlength{\leftmargini}{2em}			
			\begin{itemize}
				\item 원격저장소 에서 로컬저장소로 내려 받기 
				\item 
				\item 
				\item 
				\item 
				\item 
			\end{itemize}
			\end{block}						
		\end{frame}			

	%	---------------------------------------------------------- page		15
		\begin{frame} [t,plain]
			\begin{block} {init }
			\setlength{\leftmargini}{2em}			
			\begin{itemize}
				\item 
				\item 
				\item 
				\item 
				\item 
				\item 
			\end{itemize}
			\end{block}						
		\end{frame}			

	%	---------------------------------------------------------- page		16
		\begin{frame} [t,plain]
			\begin{block} {add}
			\setlength{\leftmargini}{2em}			
			\begin{itemize}
				\item 
				\item 
				\item 
				\item 
				\item 
				\item 
			\end{itemize}
			\end{block}						
		\end{frame}			

	%	---------------------------------------------------------- page		17
		\begin{frame} [t,plain]
			\begin{block} {log}
			\setlength{\leftmargini}{2em}			
			\begin{itemize}
				\item 
				\item 
				\item 
				\item 
				\item 
				\item 
			\end{itemize}
			\end{block}						
		\end{frame}		

	%	---------------------------------------------------------- page		18
		\begin{frame} [t,plain]
			\begin{block} {checkout}
			\setlength{\leftmargini}{2em}			
			\begin{itemize}
				\item 
				\item 
				\item 
				\item 
				\item 
				\item 
			\end{itemize}
			\end{block}						
		\end{frame}			
	
	%	---------------------------------------------------------- page		19
		\begin{frame} [t,plain]
			\begin{block} {}
			\setlength{\leftmargini}{2em}			
			\begin{itemize}
				\item 
				\item 
				\item 
				\item 
				\item 
				\item 
			\end{itemize}
			\end{block}						
		\end{frame}			

	%	---------------------------------------------------------- page		20
		\begin{frame} [t,plain]
			\begin{block} {}
			\setlength{\leftmargini}{2em}			
			\begin{itemize}
				\item 
				\item 
				\item 
				\item 
				\item 
				\item 
			\end{itemize}
			\end{block}						
		\end{frame}			


	%	========================================================== 깃 브랜치
	%												%
	%												%
	%												%
	%	========================================================== page		1

		\part{깃 브랜치 }
		\frame{\partpage}


\label{part1} 	%  깃 브랜치


	%	---------------------------------------------------------- page		2 목차
		\begin{frame} [plain]{목차}
		\tableofcontents%
		\end{frame}

	%	---------------------------------------------------------- 깃 저장소 만들기
	%		Frame
	%	----------------------------------------------------------
		\section{브랜치}
	%	---------------------------------------------------------- page		3
		\begin{frame} [t,plain]
			\begin{block} {브랜치}
			\setlength{\leftmargini}{2em}			
			\begin{itemize}
				\item 
				\item 
				\item 
			\end{itemize}
			\end{block}						
		\end{frame}						

		\section{Master 브랜치}
	%	---------------------------------------------------------- page		4
		\begin{frame} [t,plain]
			\begin{block} {Master 브랜치}
			\setlength{\leftmargini}{2em}			
			\begin{itemize}
				\item 
				\item 
				\item 
			\end{itemize}
			\end{block}						
		\end{frame}						

		\section{브랜치 만들기}
	%	---------------------------------------------------------- page		5
		\begin{frame} [t,plain]
			\begin{block} {브랜치 만들기}
			\setlength{\leftmargini}{2em}			
			\begin{itemize}
				\item 
				\item 
				\item 
			\end{itemize}
			\end{block}						
		\end{frame}						


		
		\section{브랜치 만들기 : 종류 }
	%	---------------------------------------------------------- page		6
		\begin{frame} [t,plain]
			\begin{block} {브랜치 만들기 : 종류 }
			\setlength{\leftmargini}{2em}			
			\begin{itemize}
				\item 
				\item 
				\item 
			\end{itemize}
			\end{block}						
		\end{frame}						

		\section{통합 브랜치 Integration } 
	%	---------------------------------------------------------- page		7
		\begin{frame} [t,plain]
			\begin{block} {통합 브랜치 Integration } 
			\setlength{\leftmargini}{2em}			
			\begin{itemize}
				\item 
				\item 
				\item 
			\end{itemize}
			\end{block}						
		\end{frame}						

		\section{토픽 브랜치 Topic }
	%	---------------------------------------------------------- page		8
		\begin{frame} [t,plain]
			\begin{block} {토픽 브랜치 Topic }
			\setlength{\leftmargini}{2em}			
			\begin{itemize}
				\item 
				\item 
				\item 
			\end{itemize}
			\end{block}						
		\end{frame}						

		\section{브랜치 전환하기}
	%	---------------------------------------------------------- page		9
		\begin{frame} [t,plain]
			\begin{block} {브랜치 전환하기}
			\setlength{\leftmargini}{2em}			
			\begin{itemize}
				\item 
				\item 
				\item 
			\end{itemize}
			\end{block}						
		\end{frame}						

		\section{브랜치 통합하기}
	%	---------------------------------------------------------- page		10
		\begin{frame} [t,plain]
			\begin{block} {브랜치 통합하기}
			\setlength{\leftmargini}{2em}			
			\begin{itemize}
				\item 
				\item 
				\item 
			\end{itemize}
			\end{block}						
		\end{frame}						

		\section{토픽브랜치 와 통합브랜치에서의 작업흐름 파악하기}
	%	---------------------------------------------------------- page		11
		\begin{frame} [t,plain]
			\begin{block} {토픽브랜치 와 통합브랜치에서의 작업흐름 파악하기}
			\setlength{\leftmargini}{2em}			
			\begin{itemize}
				\item 
				\item 
				\item 
			\end{itemize}
			\end{block}						
		\end{frame}						

	%	---------------------------------------------------------- page		12
		\begin{frame} [t,plain]
		\end{frame}						

	%	========================================================== 깃으로 버전 관리하기
	%												%
	%												%
	%												%
	%	========================================================== page		1

		\part{깃으로 버전 관리하기 \\중앙도서관 39 page }
		\frame{\partpage}


\label{part1} 	%  깃으로 버전 관리하기


	%	---------------------------------------------------------- page		2 목차
		\begin{frame} [plain]{목차}
		\tableofcontents%

중앙도서관 39 page 

		\end{frame}
		

			

	%	---------------------------------------------------------- 깃 저장소 만들기
	%		Frame
	%	----------------------------------------------------------
		\section{깃 저장소 만들기}

	%	---------------------------------------------------------- page		3
		\begin{frame} [t,plain]
		\frametitle{깃 저장소 만들기}

			\begin{block} {깃 저장소 만들기}
			\setlength{\leftmargini}{2em}			
			\begin{itemize}
				\item 저장소를 만들고 싶은 디렉토리로 이동해서 깃을 초기화하면
그때 부터 해당 디렉터리에 있는 파일들을 버전관리 할수 있습니다

			\end{itemize}
			\end{block}						

			\begin{block} {해당 디렉토리로 이동}
			\setlength{\leftmargini}{2em}			
			\begin{itemize}
				\item 탐색기에서 해당 디렉토리로 이동
				\item  Git Bash Here
			\end{itemize}
			\end{block}						


			\begin{block} {깃 초기화 하기 - git init}
			\setlength{\leftmargini}{2em}			
			\begin{itemize}
				\item git init
			\end{itemize}
			\end{block}						

		\end{frame}						


	%	---------------------------------------------------------- 버전 만들기
	%		Frame
	%	----------------------------------------------------------
		\section{버전 만들기}

	%	---------------------------------------------------------- page		4 
		\begin{frame} [t,plain]
		\frametitle{버전 만들기}

			\begin{block} {깃에서 버전이란}
			\setlength{\leftmargini}{2em}			
			\begin{itemize}
				\item 
			\end{itemize}
			\end{block}	


			\begin{block} {깃에서 버전이란}
			\setlength{\leftmargini}{2em}			
			\begin{itemize}
				\item 작업트리
				\item 스테이지
				\item 저장소
			\end{itemize}
			\end{block}	

				
		\end{frame}						


	%	---------------------------------------------------------- page		5 
		\begin{frame} [t,plain]
%		\frametitle{버전 만들기}

			\begin{block} {버전 만들기}
문서를 수정 저장하면 그 파일은 작업트리에 있게된다.
그리고 수정한 파일은 버전으로 만들고 싶을때 스테이지에 넣습니다.
다른 파일도 수정한 뒤 버전으로 만들겠다면 스테이지에 넣어 둡니다.
파일 수정을 끝내고 스테이지에 다 넣었다면 버전을 만들기 위해 깃에서 커밋명령을 내립니다.
커밋 명령을 내리면 새로운 버전이 생성되면서 스테이지에 대기하던 파일이 모두 저장소에 저장 됩니다.
			\end{block}	



					
		\end{frame}						

	%	---------------------------------------------------------- 커밋 내용 확인하기
	%		Frame
	%	----------------------------------------------------------
		\section{커밋 내용 확인하기}


	%	---------------------------------------------------------- page		6
		\begin{frame} [t,plain]
		\frametitle{커밋 내용 확인하기}
			\begin{block} {커밋 내용 확인하기 }
			\setlength{\leftmargini}{2em}			
			\begin{itemize}
				\item 
			\end{itemize}
			\end{block}						
		\end{frame}						

	%	---------------------------------------------------------- 버전 만드는 단계마다 파일 상태 알아보기
	%		Frame
	%	----------------------------------------------------------
		\section{버전 만드는 단계마다 파일 상태 알아보기}

	%	---------------------------------------------------------- page		7
		\begin{frame} [t,plain]
		\frametitle{버전 만드는 단계마다 파일 상태 알아보기}
			\begin{block} {버전 만드는 단계마다 파일 상태 알아보기}
			\setlength{\leftmargini}{2em}			
			\begin{itemize}
				\item 
			\end{itemize}
			\end{block}						
		\end{frame}						

	%	---------------------------------------------------------- 작업 되돌리기
	%		Frame
	%	----------------------------------------------------------
		\section{작업 되돌이기}

	%	---------------------------------------------------------- page		8
		\begin{frame} [t,plain]
		\frametitle{작업 되돌이기}
			\begin{block} {작업 되돌이기}
			\setlength{\leftmargini}{2em}			
			\begin{itemize}
				\item 
			\end{itemize}
			\end{block}						
		\end{frame}						




	%	============================================================================== 기타
		\part{깃과 브렌치 \\ 중앙도서관 p85 }
		\frame{\partpage}
		
\label{part3} 	%  키타 현황

		\begin{frame} [plain]{목차}
		\tableofcontents%
		\end{frame}

	%	---------------------------------------------------------- 브랜치란
	%		Frame
	%	----------------------------------------------------------
		\section{ 브랜치란}

		\begin{frame} [t,plain]
		\frametitle{브랜치란 }
			\begin{block} {브랜치란 }
			\setlength{\leftmargini}{1em}			
			\begin{itemize}
				\item 	
				\item 	
				\item 	
			\end{itemize}
			\end{block}						
		\end{frame}						




	%	---------------------------------------------------------- 브랜치 만들기
	%		Frame
	%	----------------------------------------------------------
		\section{ 브랜치 만들기}

		\begin{frame} [t,plain]
		\frametitle{브랜치 만들기 }
			\begin{block} {브랜치 만들기 }
			\setlength{\leftmargini}{1em}			
			\begin{itemize}
				\item 	
				\item 	
				\item 	
			\end{itemize}
			\end{block}						
		\end{frame}						

	%	---------------------------------------------------------- 브렌치 정보 확인하기
	%		Frame
	%	----------------------------------------------------------
		\section{ 브렌치 정보 확인하기}

		\begin{frame} [t,plain]
		\frametitle{브렌치 정보 확인하기 }
			\begin{block} {브렌치 정보 확인하기 }
			\setlength{\leftmargini}{1em}			
			\begin{itemize}
				\item 	
				\item 	
				\item 	
			\end{itemize}
			\end{block}						
		\end{frame}						
	%	---------------------------------------------------------- 브랜치 병합하기

	%		Frame
	%	----------------------------------------------------------
		\section{ 브랜치 병합하기}

		\begin{frame} [t,plain]
		\frametitle{브랜치 병합하기 }
			\begin{block} {브랜치 병합하기 }
			\setlength{\leftmargini}{1em}			
			\begin{itemize}
				\item 	
				\item 	
				\item 	
			\end{itemize}
			\end{block}						
		\end{frame}						

	%	---------------------------------------------------------- 브랜치 관리하기
	%		Frame
	%	----------------------------------------------------------
		\section{ 브랜치 관리하기}

		\begin{frame} [t,plain]
		\frametitle{브랜치 관리하기 }
			\begin{block} { 브랜치 관리하기}
			\setlength{\leftmargini}{1em}			
			\begin{itemize}
				\item 	
				\item 	
				\item 	
			\end{itemize}
			\end{block}						
		\end{frame}						

	%	----------------------------------------------------------
		\begin{frame} [t,plain]
		\end{frame}						

	%	========================================================== 깃허브로 백업하기
	%
	%
	%
	%	---------------------------------------------------------- 	page 	1
		\part{깃허브로 백업하기 \\ 중앙도서관 P 132}
		\frame{\partpage}

\label{part1} 	%  차 시간 정리


	%	---------------------------------------------------------- 	page 	2
		\begin{frame} [plain]{목차}
		\tableofcontents%
		\end{frame}

	%	---------------------------------------------------------- 	page 	3
		\begin{frame} [t,plain]
			\begin{block} {원격저장소와 깃허브}
			\setlength{\leftmargini}{1em}			
			\begin{itemize}
				\item 	
				\item 	
				\item 	
			\end{itemize}
			\end{block}						
		\end{frame}						

	%	---------------------------------------------------------- 	page 	4
		\begin{frame} [t,plain]
			\begin{block} {깃허브 시작하기}
			\setlength{\leftmargini}{1em}			
			\begin{itemize}
				\item 	
				\item 	
				\item 	
			\end{itemize}
			\end{block}						
		\end{frame}						

	%	---------------------------------------------------------- 	page 	5
		\begin{frame} [t,plain]
			\begin{block} {지역저장소를 원격저장소에 연결하기}
			\setlength{\leftmargini}{1em}			
			\begin{itemize}
				\item 	
				\item 	
				\item 	
			\end{itemize}
			\end{block}						
		\end{frame}						

	%	---------------------------------------------------------- 	page 	6
		\begin{frame} [t,plain]
			\begin{block} {원격저장소에 올리기 및 내려받기}
			\setlength{\leftmargini}{1em}			
			\begin{itemize}
				\item 	
				\item 	
				\item 	
			\end{itemize}
			\end{block}						
		\end{frame}						

	%	---------------------------------------------------------- 	page 	7
		\begin{frame} [t,plain]
			\begin{block} {깃허브에 SSH원격 접속하기 }
			\setlength{\leftmargini}{1em}			
			\begin{itemize}
				\item 	
				\item 	
				\item 	
			\end{itemize}
			\end{block}						
		\end{frame}						


	%	---------------------------------------------------------- 	page 	8
		\begin{frame} [t,plain]
		\end{frame}						


	%	========================================================== 깃허브로 협업하기
	%
	%
	%	---------------------------------------------------------- 	page 	1
		\part{깃허브로 협업하기 \\중앙도서관 167 page }
		\frame{\partpage}

\label{part1} 	%  깃허브로 협업하기

	%	---------------------------------------------------------- 	page 	2
		\begin{frame} [plain]{목차}
		\tableofcontents%
		\end{frame}


	%	---------------------------------------------------------- 	page 	3
		\begin{frame} [t,plain]
			\begin{block} {여러 컴퓨터에서 원격 저장소 함께 사용하기}
			\setlength{\leftmargini}{1em}			
			\begin{itemize}
				\item 	
				\item 	
				\item 	
			\end{itemize}
			\end{block}						
		\end{frame}						

	%	---------------------------------------------------------- 	page 	4
		\begin{frame} [t,plain]
			\begin{block} {원격 브랜치 정보 가져오기}
			\setlength{\leftmargini}{1em}			
			\begin{itemize}
				\item 	
				\item 	
				\item 	
			\end{itemize}
			\end{block}						
		\end{frame}						

	%	---------------------------------------------------------- 	page 	5
		\begin{frame} [t,plain]
			\begin{block} {협업의 기본 알아보기}
			\setlength{\leftmargini}{1em}			
			\begin{itemize}
				\item 	
				\item 	
				\item 	
			\end{itemize}
			\end{block}						
		\end{frame}						

	%	---------------------------------------------------------- 	page 	6
		\begin{frame} [t,plain]
			\begin{block} {협업에서 브랜치 사용하기}
			\setlength{\leftmargini}{1em}			
			\begin{itemize}
				\item 	
				\item 	
				\item 	
			\end{itemize}
			\end{block}						
		\end{frame}						


	%	---------------------------------------------------------- 	page 	7
		\begin{frame} [t,plain]
		\end{frame}						

	%	---------------------------------------------------------- 	page 	8
		\begin{frame} [t,plain]
		\end{frame}						



	%	========================================================== 기본용어
		\part{기본 용어}
		\frame{\partpage}

\label{part1} 	%  기본 용어

		\begin{frame} [plain]{목차}
		\tableofcontents%
		\end{frame}

	%	---------------------------------------------------------- 3
		\begin{frame} [t,plain]

			\begin{block} { 커멘드 라인 }
			\setlength{\leftmargini}{1em}			
			\begin{itemize}
				\item 	
				\item 	
				\item 	
			\end{itemize}
			\end{block}						

		\end{frame}						

	%	---------------------------------------------------------- 4
		\begin{frame} [t,plain]

			\begin{block} { 저장소 re po si to ry }
			\setlength{\leftmargini}{1em}			
			\begin{itemize}
				\item 	
				\item 	
				\item 	
			\end{itemize}
			\end{block}						

		\end{frame}						

	%	---------------------------------------------------------- 5 버전관리
		\begin{frame} [t,plain]

			\begin{block} { 버전 관리 }
			\setlength{\leftmargini}{1em}			
			\begin{itemize}
				\item 	
				\item 	
				\item 	
			\end{itemize}
			\end{block}						

		\end{frame}						

	%	---------------------------------------------------------- 6 	커밋 com mit
		\begin{frame} [t,plain]

			\begin{block} { 커밋 com mit }
			\setlength{\leftmargini}{1em}			
			\begin{itemize}
				\item 	
				\item 	
				\item 	
			\end{itemize}
			\end{block}						

		\end{frame}						



	%	---------------------------------------------------------- 7 	브랜치 bran ch
		\begin{frame} [t,plain]

			\begin{block} {브랜치 bran ch }
			\setlength{\leftmargini}{1em}			
			\begin{itemize}
				\item 	
				\item 	
				\item 	
			\end{itemize}
			\end{block}						

		\end{frame}						

	%	---------------------------------------------------------- 8
		\begin{frame} [t,plain]

			\begin{block} { 저장소 re po si to ry }
			\setlength{\leftmargini}{1em}			
			\begin{itemize}
				\item 	
				\item 	
				\item 	
			\end{itemize}
			\end{block}						

		\end{frame}						


	%	========================================================== 주요 명령어
		\part{주요 명령어}
		\frame{\partpage}

\label{part1} 	%  주요 명령어

		\begin{frame} [plain]{목차}
		\tableofcontents%
		\end{frame}

	%	----------------------------------------------------------
		\begin{frame} [t,plain]
		\end{frame}						

	%	----------------------------------------------------------
		\begin{frame} [t,plain]
		\end{frame}						


	%	 ---------------------------------------------------------- 					1	init			
		\begin{frame} [t,plain]									
											
			\begin{block} {		init		}				
			\setlength{\leftmargini}{1em}								
			\begin{itemize}								
				\item							
				\item							
				\item							
			\end{itemize}								
			\end{block}								
											
		\end{frame}									

	%	 ---------------------------------------------------------- 					2	con fig			
		\begin{frame} [t,plain]									
											
			\begin{block} {		con fig		}				
			\setlength{\leftmargini}{1em}								
			\begin{itemize}								
				\item							
				\item							
				\item							
			\end{itemize}								
			\end{block}								
											
		\end{frame}									

	%	 ---------------------------------------------------------- 					3	help			
		\begin{frame} [t,plain]									
											
			\begin{block} {		help		}				
			\setlength{\leftmargini}{1em}								
			\begin{itemize}								
				\item							
				\item							
				\item							
			\end{itemize}								
			\end{block}								
											
		\end{frame}									

	%	 ---------------------------------------------------------- 					4	sta rtus			
		\begin{frame} [t,plain]									
											
			\begin{block} {		sta rtus		}				
			\setlength{\leftmargini}{1em}								
			\begin{itemize}								
				\item							
				\item							
				\item							
			\end{itemize}								
			\end{block}								
											
		\end{frame}									

	%	 ---------------------------------------------------------- 					5	add			
		\begin{frame} [t,plain]									
											
			\begin{block} {		add		}				
			\setlength{\leftmargini}{1em}								
			\begin{itemize}								
				\item							
				\item							
				\item							
			\end{itemize}								
			\end{block}								
											
		\end{frame}									


	%	 ---------------------------------------------------------- 					6	commit			
		\begin{frame} [t,plain]									
											
			\begin{block} {		commit		}				
			\setlength{\leftmargini}{1em}								
			\begin{itemize}								
				\item							
				\item							
				\item							
			\end{itemize}								
			\end{block}								
											
		\end{frame}									

	%	 ---------------------------------------------------------- 					7	branch			
		\begin{frame} [t,plain]									
											
			\begin{block} {		branch		}				
			\setlength{\leftmargini}{1em}								
			\begin{itemize}								
				\item							
				\item							
				\item							
			\end{itemize}								
			\end{block}								
											
		\end{frame}									

	%	 ---------------------------------------------------------- 					8	check out			
		\begin{frame} [t,plain]									
											
			\begin{block} {		check out		}				
			\setlength{\leftmargini}{1em}								
			\begin{itemize}								
				\item							
				\item							
				\item							
			\end{itemize}								
			\end{block}								
											
		\end{frame}									

	%	 ---------------------------------------------------------- 					9	mer ge			
		\begin{frame} [t,plain]									
											
			\begin{block} {		mer ge		}				
			\setlength{\leftmargini}{1em}								
			\begin{itemize}								
				\item							
				\item							
				\item							
			\end{itemize}								
			\end{block}								
											
		\end{frame}									

	%	 ---------------------------------------------------------- 					10	push			
		\begin{frame} [t,plain]									
											
			\begin{block} {		push		}				
			\setlength{\leftmargini}{1em}								
			\begin{itemize}								
				\item							
				\item							
				\item							
			\end{itemize}								
			\end{block}								
											
		\end{frame}									


	%	 ---------------------------------------------------------- 					11	pull			
		\begin{frame} [t,plain]									
											
			\begin{block} {		pull		}				
			\setlength{\leftmargini}{1em}								
			\begin{itemize}								
				\item							
				\item							
				\item							
			\end{itemize}								
			\end{block}								
											
		\end{frame}									


	%	 ---------------------------------------------------------- 					12	빈페이지
		\begin{frame} [t,plain]
		\end{frame}						


% ------------------------------------------------------------------------------
% End document
% ------------------------------------------------------------------------------





\end{document}


