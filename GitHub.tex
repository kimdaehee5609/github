%	-------------------------------------------------------------------------------
% 
%			2020년 
%			12월 
%			26일 
%			토요일
%			첫작업
%
%
%
%
%
%
%
%	-------------------------------------------------------------------------------

%	\documentclass[12pt, a3paper, oneside]{book}
	\documentclass[12pt, a4paper, oneside]{book}
%	\documentclass[12pt, a4paper, landscape, oneside]{book}

		% --------------------------------- 페이지 스타일 지정
		\usepackage{geometry}
%		\geometry{landscape=true	}
		\geometry{top 			=10em}
		\geometry{bottom			=10em}
		\geometry{left			=8em}
		\geometry{right			=8em}
		\geometry{headheight		=4em} % 머리말 설치 높이
		\geometry{headsep		=2em} % 머리말의 본문과의 띠우기 크기
		\geometry{footskip		=4em} % 꼬리말의 본문과의 띠우기 크기
% 		\geometry{showframe}
	
%		paperwidth 	= left + width + right (1)
%		paperheight 	= top + height + bottom (2)
%		width 		= textwidth (+ marginparsep + marginparwidth) (3)
%		height 		= textheight (+ headheight + headsep + footskip) (4)	



		%	=================================================================== package
		%	package
		%	===================================================================
%			\usepackage[hangul]{kotex}				% 한글 사용
			\usepackage{kotex}					% 한글 사용
			\usepackage[unicode]{hyperref}			% 한글 하이퍼링크 사용

		% ------------------------------ 수학 수식
			\usepackage{amssymb,amsfonts,amsmath}	% 수학 수식 사용
			\usepackage{mathtools}				% amsmath 확장판

			\usepackage{scrextend}				% 
		

		% ------------------------------ LIST
			\usepackage{enumerate}			%
			\usepackage{enumitem}			%
			\usepackage{tablists}				%	수학문제의 보기 등을 표현하는데 사용
										%	tabenum


		% ------------------------------ table 
			\usepackage{longtable}			%
			\usepackage{tabularx}			%
			\usepackage{tabu}				%




		% ------------------------------ 
			\usepackage{setspace}			%
			\usepackage{booktabs}		% table
			\usepackage{color}			%
			\usepackage{multirow}			%
			\usepackage{boxedminipage}	% 미니 페이지
			\usepackage[pdftex]{graphicx}	% 그림 사용
			\usepackage[final]{pdfpages}		% pdf 사용
			\usepackage{framed}			% pdf 사용

			
			\usepackage{fix-cm}	
			\usepackage[english]{babel}
	
		%	======================================================================================= package
		% 	tikz package
		% 	
		% 	--------------------------------- 	
			\usepackage{tikz}%
			\usetikzlibrary{arrows,positioning,shapes}
			\usetikzlibrary{mindmap}			
			

		% --------------------------------- 	page
			\usepackage{afterpage}		% 다음페이지가 나온면 어떻게 하라는 명령 정의 패키지
%			\usepackage{fullpage}			% 잘못 사용하면 다 흐트러짐 주의해서 사용
%			\usepackage{pdflscape}		% 
			\usepackage{lscape}			%	 


			\usepackage{blindtext}
	
		% --------------------------------- font 사용
			\usepackage{pifont}				%
			\usepackage{textcomp}
			\usepackage{gensymb}
			\usepackage{marvosym}



		% Package --------------------------------- 

			\usepackage{tablists}				%


		% Package --------------------------------- 
			\usepackage[framemethod=TikZ]{mdframed}				% md framed package
			\usepackage{smartdiagram}								% smart diagram package

			\newcounter{theo}[section]\setcounter{theo}{0}
			\renewcommand{\thetheo}{\arabic{section}.\arabic{theo}}
%			\newenvironment{name}[args]{begin_def}{end_def}

 
  


		% Package ---------------------------------    연습문제 

			\usepackage{exsheets}				%

			\SetupExSheets{solution/print=true}
			\SetupExSheets{question/type=exam}
			\SetupExSheets[points]{name=point,name-plural=points}


		% --------------------------------- 페이지 스타일 지정

		\usepackage[Sonny]		{fncychap}

			\makeatletter
			\ChNameVar	{\Large\bf}
			\ChNumVar	{\Huge\bf}
			\ChTitleVar		{\Large\bf}
			\ChRuleWidth	{0.5pt}
			\makeatother

%		\usepackage[Lenny]		{fncychap}
%		\usepackage[Glenn]		{fncychap}
%		\usepackage[Conny]		{fncychap}
%		\usepackage[Rejne]		{fncychap}
%		\usepackage[Bjarne]	{fncychap}
%		\usepackage[Bjornstrup]{fncychap}

		\usepackage{fancyhdr}
		\pagestyle{fancy}
		\fancyhead{} % clear all fields
		\fancyhead[LO]{\footnotesize \leftmark}
		\fancyhead[RE]{\footnotesize \leftmark}
		\fancyfoot{} % clear all fields
		\fancyfoot[LE,RO]{\large \thepage}
		%\fancyfoot[CO,CE]{\empty}
		\renewcommand{\headrulewidth}{1.0pt}
		\renewcommand{\footrulewidth}{0.4pt}
	
	
	
		%	--------------------------------------------------------------------------------------- 
		% 	tritlesec package
		% 	
		% 	
		% 	------------------------------------------------------------------ section 스타일 지정
	
			\usepackage{titlesec}
		
		% 	----------------------------------------------------------------- section 글자 모양 설정
			\titleformat*{\section}					{\large\bfseries}
			\titleformat*{\subsection}				{\normalsize\bfseries}
			\titleformat*{\subsubsection}			{\normalsize\bfseries}
			\titleformat*{\paragraph}				{\normalsize\bfseries}
			\titleformat*{\subparagraph}				{\normalsize\bfseries}
	
		% 	----------------------------------------------------------------- section 번호 설정
			\renewcommand{\thepart}				{\arabic{part}.}
			\renewcommand{\thesection}				{\arabic{section}.}
			\renewcommand{\thesubsection}			{\thesection\arabic{subsection}.}
			\renewcommand{\thesubsubsection}		{\thesubsection\arabic{subsubsection}}
			\renewcommand\theparagraph 			{$\blacksquare$ \hspace{3pt}}

		% 	----------------------------------------------------------------- section 페이지 나누기 설정
			\let\stdsection\section
			\renewcommand\section{\newpage\stdsection}



		%	--------------------------------------------------------------------------------------- 
		% 	\titlespacing*{commandi} {left} {before-sep} {after-sep} [right-sep]		
		% 	left
		%	before-sep		:  수직 전 간격
		% 	after-sep	 	:  수직으로 후 간격
		%	right-sep

			\titlespacing*{\section} 			{0pt}{1.0em}{1.0em}
			\titlespacing*{\subsection}	  		{0ex}{1.0em}{1.0em}
			\titlespacing*{\subsubsection}		{0ex}{1.0em}{1.0em}
			\titlespacing*{\paragraph}			{0em}{1.5em}{1.0em}
			\titlespacing*{\subparagraph}		{4em}{1.0em}{1.0em}
	
		%	\titlespacing*{\section} 			{0pt}{0.0\baselineskip}{0.0\baselineskip}
		%	\titlespacing*{\subsection}	  		{0ex}{0.0\baselineskip}{0.0\baselineskip}
		%	\titlespacing*{\subsubsection}		{6ex}{0.0\baselineskip}{0.0\baselineskip}
		%	\titlespacing*{\paragraph}			{6pt}{0.0\baselineskip}{0.0\baselineskip}
	

		% --------------------------------- recommend		섹션별 페이지 상단 여백
			\newcommand{\SectionMargin}				{\newpage  \null \vskip 2cm}
			\newcommand{\SubSectionMargin}			{\newpage  \null \vskip 2cm}
			\newcommand{\SubSubSectionMargin}		{\newpage  \null \vskip 2cm}


		%	--------------------------------------------------------------------------------------- 
		%
		% 	toc 설정  - table of contents
		% 	
		% 	
		% 	----------------------------------------------------------------  문서 기본 사항 설정
			\setcounter{secnumdepth}{0} 		% 문단 번호 깊이 chapter
			\setcounter{secnumdepth}{1} 		% 문단 번호 깊이 section
			\setcounter{secnumdepth}{2} 		% 문단 번호 깊이 subsection
			\setcounter{secnumdepth}{3} 		% 문단 번호 깊이 subsubsec
			\setcounter{secnumdepth}{4} 		% 문단 번호 깊이 paragraph
			\setcounter{secnumdepth}{5} 		% 문단 번호 깊이 subparagraph
			\setcounter{tocdepth}{2} 			% 문단 번호 깊이 - 목차 출력시 출력 범위

			\setlength{\parindent}{0cm} 		% 문서 들여 쓰기를 하지 않는다.


		%	--------------------------------------------------------------------------------------- 
		% 	mini toc 설정
		% 	
		% 	
		% 	--------------------------------------------------------- 장의 목차  minitoc package
			\usepackage{minitoc}

%			\setcounter{minitocdepth}{0}    	% 	chapter 
			\setcounter{minitocdepth}{1}    	%  	secton
%			\setcounter{minitocdepth}{2}    	%  	subsection
%			\setcounter{minitocdepth}{3}    	%  	sub sub section
%			\setcounter{minitocdepth}{4}    	%  	paragraph
%			\setcounter{minitocdepth}{5}    	%  	sub paragraph
%			\setlength{\mtcindent}{12pt} 		%	default 24pt
			\setlength{\mtcindent}{24pt} 		% 	default 24pt

		% 	--------------------------------------------------------- part toc
		%	\setcounter{parttocdepth}{2} 		%  default
			\setcounter{parttocdepth}{0}
		%	\setlength{\ptcindent}{0em}		%  default  목차 내용 들여 쓰기
			\setlength{\ptcindent}{0em}         


		% 	--------------------------------------------------------- section toc
			\renewcommand{\ptcfont}{\normalsize\rm} 		%  default
			\renewcommand{\ptcCfont}{\normalsize\bf} 	%  default
			\renewcommand{\ptcSfont}{\normalsize\rm} 	%  default

		% 	--------------------------------------------------------- section toc
%			\setcounter{sectiondepth}{2}
%			\setcounter{\secttocdepth}{2}				% default
%			\setlength{\stcindent}{24pt}				% default
%			\renewcommand{\stcfont}{f\small\rm}		% default
%			\renewcommand{\stcSSfont}{f\small\bf} 	% default


		%	=======================================================================================
		% 	tocloft package
		% 	
		% 	------------------------------------------ 목차의 목차 번호와 목차 사이의 간격 조정
			\usepackage{tocloft}

		% 	------------------------------------------ 목차의 내어쓰기 즉 왼쪽 마진 설정
			\setlength{\cftsecindent}{2em}			%  section

		% 	------------------------------------------ 목차의 목차 번호와 목차 사이의 간격 조정
			\setlength{\cftsecnumwidth}{2em}		%  section





		%	=======================================================================================
		% 	flowchart  package
		% 	
		% 	------------------------------------------ 목차의 목차 번호와 목차 사이의 간격 조정
			\usepackage{flowchart}
			\usetikzlibrary{arrows}

		%	======================================================================================= 
		% 	t color box
		% 	
		% 	------------------------------------------ 상자안에 강조 문자
			\usepackage{tcolorbox}
%			\usepackage[listings]{tcolorbox}
			\tcbuselibrary{raster}
			\tcbuselibrary{breakable}

		%	=======================================================================================  코드 입력
			\usepackage[]{quoting}
			\usepackage{csquotes}		%
			\usepackage{epigraph}		%
			\usepackage{epigraph}		%

			\usepackage{epigraph}		% 	 \begin{savequote}[45mm]
										%		\qauthor{Shakespeare, Macbeth}
										%		Cookies! Give me some cookies!
										%		\qauthor{Cookie Monster}
										%	\end{savequote}

										%	\begin{quote}
										%	\lipsum[1-2]
										%	\end{quote}

										%	\begin{quotation}
										%	\lipsum[1-2]
										%	\end{quotation}

										%	\begin{quoting}
										%	\lipsum[1-2]
										%	\end{quoting}

										%	\begin{verbatim}
										%		core.ignorecase=true
										%	\end{verbatim}



		%	=======================================================================================
		% 		makeindex package
		% 	
		% 	------------------------------------------ 목차의 목차 번호와 목차 사이의 간격 조정
%			\usepackage{makeindex}
%			\usepackage{makeidy}



		%	=======================================================================================
		% 		각주와 미주
		% 	

		\usepackage{endnotes} %미주 사용


		%	=======================================================================================
		% 	줄 간격 설정
		% 	
		% 	
		% 	--------------------------------- 	줄간격 설정
			\doublespace
%			\onehalfspace
%			\singlespace
		
		

	% 	============================================================================== itemi Global setting

	
		%	-------------------------------------------------------------------------------
		%		Vertical spacing
		%	-------------------------------------------------------------------------------
			\setlist[itemize]{topsep=0.0em}			% 상단의 여유치
			\setlist[itemize]{partopsep=0.0em}			% 
			\setlist[itemize]{parsep=0.0em}			% 
%			\setlist[itemize]{itemsep=0.0em}			% 
			\setlist[itemize]{noitemsep}				% 
			
		%	-------------------------------------------------------------------------------
		%		Horizontal spacing
		%	-------------------------------------------------------------------------------
			\setlist[itemize]{labelwidth=1em}			%  라벨의 표시 폭
			\setlist[itemize]{leftmargin=8em}			%  본문 까지의 왼쪽 여백  - 4em
			\setlist[itemize]{labelsep=3em} 			%  본문에서 라벨까지의 거리 -  3em
			\setlist[itemize]{rightmargin=0em}			% 오른쪽 여백  - 4em
			\setlist[itemize]{itemindent=0em} 			% 점 내민 거리 label sep 과 같은면 점위치 까지 내민다
			\setlist[itemize]{listparindent=3em}		% 본문 드려쓰기 간격
	
	
			\setlist[itemize]{ topsep=0.0em, 			%  상단의 여유치
						partopsep=0.0em, 		%  
						parsep=0.0em, 
						itemsep=0.0em, 
						labelwidth=1em, 
						leftmargin=2.5em,
						labelsep=2em,			%  본문에서 라벨 까지의 거리
						rightmargin=0em,		% 오른쪽 여백  - 4em
						itemindent=0em, 		% 점 내민 거리 label sep 과 같은면 점위치 까지 내민다
						listparindent=0em}		% 본문 드려쓰기 간격
	
%			\begin{itemize}
	
		%	-------------------------------------------------------------------------------
		%		Label
		%	-------------------------------------------------------------------------------
			\renewcommand{\labelitemi}{$\bullet$}
			\renewcommand{\labelitemii}{$\bullet$}
%			\renewcommand{\labelitemii}{$\cdot$}
			\renewcommand{\labelitemiii}{$\diamond$}
			\renewcommand{\labelitemiv}{$\ast$}		
	
%			\renewcommand{\labelitemi}{$\blacksquare$}   	% 사각형 - 찬것
%			\renewcommand\labelitemii{$\square$}		% 사각형 - 빈것	
			






% ------------------------------------------------------------------------------
% Begin document (Content goes below)
% ------------------------------------------------------------------------------ 표지
	\begin{document}
	
			\dominitoc
			\doparttoc			

%			\dosecttoc
%			\dosectlof
%			\dosectlot

			\title{ Git Github  }
			\author{김대희}
			\date{ 	2020년 
					12월 
					28일
					일요일
						}
			\maketitle


			\tableofcontents 		% 목차 출력
%			\listoffigures 			% 그림 목차 출력
			\cleardoublepage
			\listoftables 			% 표 목차 출력





		\mdfdefinestyle	{con_specification} {
						outerlinewidth		=1pt			,%
						innerlinewidth		=2pt			,%
						outerlinecolor		=blue!70!black	,%
						innerlinecolor		=white 			,%
						roundcorner			=4pt			,%
						skipabove			=1em 			,%
						skipbelow			=1em 			,%
						leftmargin			=0em			,%
						rightmargin			=0em			,%
						innertopmargin		=2em 			,%
						innerbottommargin 	=2em 			,%
						innerleftmargin		=1em 			,%
						innerrightmargin		=1em 			,%
						backgroundcolor		=gray!4			,%
						frametitlerule		=true 			,%
						frametitlerulecolor	=white			,%
						frametitlebackgroundcolor=black		,%
						frametitleaboveskip=1em 			,%
						frametitlebelowskip=1em 			,%
						frametitlefontcolor=white 			,%
						}


%	개정일력
%	2020.12.26 첫 작업
%	2020.12.28 깃 업로드
%


%	================================================================== Part			개요
	\addtocontents{toc}{\protect\newpage}
	\part{ 개요  }
	\noptcrule
	\parttoc				

% ----------------------------------------------------------------------------- 	깃 시작하기
%										
% -----------------------------------------------------------------------------										
	\section{ 깃 시작하기 }


% ----------------------------------------------------------------------------- 	깃 역사		깃역사
%										
% -----------------------------------------------------------------------------										
	\section{ 깃 역사}

	\paragraph{짦게 보는 Git의 역사}

			\begin{itemize}
			\item 	2020년 리눅스 커널의 Bitkeeper 상용 DVGS 사용
			\item 	2005년 Bitkeeper의 무료 사용이 제지된다.
			\end{itemize}

							
	\paragraph{Git의 개발시 고려사항}


			\begin{itemize}
			\item 	빠른 속도
			\item 	단순한 구조
			\item 	비선층 개발 (수천개의 동시 다발적인 브랜치)
			\item 	완벽한 분산
			\item 	리눅스 커널 같은 대형프로젝트에도 유용할것 \\(속도나 데이터의 크기면에서)
			\end{itemize}



% ----------------------------------------------------------------------------- 	깃으로하고싶은것
%										
% -----------------------------------------------------------------------------										
	\section{ 깃으로 하고 싶은것}

			\begin{itemize}
			\item 	구글 드라이버 같이 파일 보관 영역으로 사용
			\item 	사무실 2층 도로공사에서 사용가능
			\end{itemize}






% ----------------------------------------------------------------------------- 	깃 설치하기
%	
% -----------------------------------------------------------------------------	
	\chapter 	{깃 설치하기}




% ----------------------------------------------------------------------------- 	깃 기본 프로그램
%	
% -----------------------------------------------------------------------------	
	\section 	{깃 기본 프로그램}




			\begin{itemize}	[
						topsep=0.0em, 
						parsep=0.0em, 
						itemsep=0em, 
						leftmargin=12.0em, 
						labelwidth=3em, 
						labelsep=3em
						] 
			\item [1.] 	알집 
			\end{itemize}


				\begin{enumerate}
				\setlength\itemsep{-1.0em}
				\item	알집
				\item	Git
				\item	GitHub Desktop
				\item	Source Tree Setup
				\item	VS Code
				\item	notepad++
				\end{enumerate}


% ----------------------------------------------------------------------------- 	깃 설치하기
%	
% -----------------------------------------------------------------------------	
	\section 	{깃 설치하기}


% ----------------------------------------------------------------------------- 	깃 세팅
%	
% -----------------------------------------------------------------------------	
	\section 	{깃 세팅}

% ----------------------------------------------------------------------------- 	깃 설치 확인
%	
% -----------------------------------------------------------------------------	
	\section 	{깃 설치 확인}


		% ------------------------------------------------- tcolorbox package
			\begin{tcolorbox}		[
									title=커밋 config 확인
								]
				\begin{verbatim}
					git config --list
					diff.astextplain.textconv=astextplain
					filter.lfs.clean=git-lfs clean -- %f
					filter.lfs.smudge=git-lfs smudge -- %f
					filter.lfs.process=git-lfs filter-process
					filter.lfs.required=true
					http.sslbackend=openssl
					http.sslcainfo=C:/Program Files/Git/mingw64/ssl/certs/ca-bundle.crt
					core.autocrlf=true
					core.fscache=true
					core.symlinks=false
					pull.rebase=false
					credential.helper=manager-core
					credential.https://dev.azure.com.usehttppath=true
					user.email=h01038395609@gmail.com
					user.name=kimdaehee5609
					filter.lfs.required=true
					filter.lfs.clean=git-lfs clean -- %f
					filter.lfs.smudge=git-lfs smudge -- %f
					filter.lfs.process=git-lfs filter-process
				\end{verbatim}
			\end{tcolorbox}

		% ------------------------------------------------- tcolorbox package
			\begin{tcolorbox}		[
									title=커밋 config 확인
								]
				\begin{verbatim}

					color.ui=true
					core.repositoryformatversion=0
					core.filemode=false
					core.bare=false
					core.logallrefupdates=true
					core.symlinks=false
					core.ignorecase=true
				\end{verbatim}
			\end{tcolorbox}







	
%	================================================================== 		깃 기본 명령어
	\addtocontents{toc}{\protect\newpage}
	\chapter {깃 기본 명령어}
	\noptcrule

%	\newpage	
%	\minitoc
%	\secttoc



% ----------------------------------------------------------------------------- 	저장소 생성
%	
% -----------------------------------------------------------------------------	
	\section 	{저장소 생성 :  git init}


% ----------------------------------------------------------------------------- 	git add
%	
% -----------------------------------------------------------------------------	
	\section 	{git add}


% ----------------------------------------------------------------------------- 	git commit
%	
% -----------------------------------------------------------------------------	
	\section 	{git commit}


% ----------------------------------------------------------------------------- 	git status
%	
% -----------------------------------------------------------------------------	
	\section 	{git status}


% ----------------------------------------------------------------------------- 	저장소 사용을 위한 명령
%	
% -----------------------------------------------------------------------------	
	\section 	{저장소 사용을 위한 명령}


%	================================================================== 		깃 설정하기
	\addtocontents{toc}{\protect\newpage}
	\chapter {깃 설정하기}
	\noptcrule

%	\newpage	
	\minitoc
%	\secttoc

% ----------------------------------------------------------------------------- 	 Git 기본 설정
%	
% -----------------------------------------------------------------------------	
	\section 	{ Git 기본 설정}


	각자의 이름과 이메일 주소를 적는다


		% ------------------------------------------------- tcolorbox package
		\begin{tcolorbox}		[
%								colback=green!5,
								colback=red!5!white,
								colframe=red!75!black,
%								colframe=green!40!black,
								title=깃 기본 설정
								]
			\$git config --glabal user.name  "kim dae hee 5609 " \\
			\$git config --glabal user.email  "h 010 3839 5609 @ gmail . com" \\
			\$git config --glabal color.ui true
		\end{tcolorbox}


		\begin{tcolorbox}
			- global : 지금 로그온한 계정에 전체 설정 \\
			- local : 현재 저장소 로컬 설정
		\end{tcolorbox}


% ----------------------------------------------------------------------------- 	 Git 설정하기
%	
% -----------------------------------------------------------------------------	
	\section 	{ Git 설정하기}

\paragraph{Git 설정하기}
지금까지 Git이 어떻게 동작하고 Git을 어떻게 사용하는지 설명했다. 이제 Git을 좀 더 쉽고 편하게 사용할 수 있도록 도와주는 도구를 살펴본다. 
이 장에서는 먼저 많이 쓰이는 설정 그리고 훅 시스템을 먼저 설명한다. 
그 후에 Git을 내게 맞추어(Customize) 본다. Git을 자신의 프로젝트에 맞추고 편하게 사용하자.



\paragraph{Git 설정하기}
시작하기에서 git config 명령을 간단히 사용했었다. git config 명령으로 제일 먼저 하게 되는 작업은 이름과 이메일 주소를 설정하는 것이다.

$ git config --global user.name "John Doe"
$ git config --global user.email johndoe@example.com

여기서는 이렇게 설정하는 것 중에서 중요한 것만 몇 가지 배운다.

우선 Git은 내장된 기본 규칙 따르지만, 설정된 것이 있으면 그에 따른다는 점을 생각해두자. Git은 먼저 /etc/gitconfig 파일을 찾는다. 
이 파일은 해당 시스템에 있는 모든 사용자와 모든 저장소에 적용되는 설정 파일이다. git config 명령에 --system 옵션을 주면 이 파일을 사용한다.

다음으로 ~/.gitconfig 파일을 찾는다. 이 파일은 해당 사용자에게만 적용되는 설정 파일이다. --global 옵션을 주면 Git은 이 파일을 사용한다.

마지막으로 현재 작업 중인 저장소의 Git 디렉토리에 있는 .git/config 파일을 찾는다. 
이 파일은 해당 저장소에만 적용된다. 
git config 명령에 --local 옵션을 적용한 것과 같다. 
(아무런 범위 옵션을 지정하지 않으면 Git은 기본적으로 --local 옵션을 적용한다)

각 설정 파일에 중복된 설정이 있으면 설명한 “순서대로” 덮어쓴다. 
예를 들어 .git/config 와 /etc/gitconfig 에 같은 설정이 들어 있다면 .git/config 에 있는 설정을 사용한다.

\paragraph{Git 설정하기: Note}
설정 파일 일반적인 텍스트파일로 쉽게 고쳐 쓸 수 있다. 보통 git config 명령을 사용하는 것이 더 편하다.

\subsection{클라이언트 설정}

설정이 영향을 미치는 대상에 따라 클라이언트 설정과 서버 설정으로 나눠볼 수 있다. 대부분은 개인작업 환경과 관련된 클라이언트 설정이다. Git에는 설정거리가 매우 많은데, 여기서는 워크플로를 관리하는 데 필요한 것과 주로 많이 사용하는 것만 설명한다. 한 번도 겪지 못할 상황에서나 유용한 옵션까지 다 포함하면 설정할 게 너무 많다. Git 버전마다 옵션이 조금씩 다른데, 아래와 같이 실행하면 설치한 버전에서 사용할 수 있는 옵션을 모두 보여준다.

\$ man git-config
어떤 옵션을 사용할 수 있는지 매우 자세히 설명하고 있다. http://git-scm.com/docs/git-config.html 페이지에서도 같은 내용을 볼 수 있다.

core.editor
Git은 편집기를 설정(\$VISUAL, \$EDITOR 변수로 설정)하지 않았거나 설정한 편집기를 찾을 수 없으면 vi 를 실행한다. 커밋할 때나 태그 메시지를 편집할 때 설정한 편집기를 실행한다. code.editor 설정으로 편집기를 설정한다.

\$ git config --global core.editor emacs
이렇게 설정하면 메시지를 편집할 때 환경변수에 설정한 편집기가 아니라 Emacs를 실행한다.

commit.template
커밋할 때 Git이 보여주는 커밋 메시지는 이 옵션에 설정한 템플릿 파일이다. 사용자 지정 커밋 템플릿 메시지가 주는 장점은 커밋 메시지를 작성할 때 일정한 스타일을 유지할 수 있다는 점이다.

예를 들어 ~/.gitmessage.txt 파일을 아래와 같이 만든다.

Subject line (try to keep under 50 characters)

Multi-line description of commit,
feel free to be detailed.

[Ticket: X]
커밋 메시지 템플릿을 보면 커밋 메시지를 작성할 때 제목은 일정 길이 이하로 짧게 하고(git log --oneline 으로 보기 좋게) 자세한 수정 내용은 한칸 공백 이후 서술하도록 하고 버그 트래킹 시스템이나 이슈 관리 시스템을 쓸 경우 이슈의 번호를 적도록 유도하고 있는 것을 볼 수 있다.

이 파일을 commit.template 에 설정하면 Git은 git commit 명령이 실행하는 편집기에 이 메시지를 기본으로 넣어준다.

\$ git config --global commit.template ~/.gitmessage.txt
\$ git commit
그러면 커밋할 때 아래와 같은 메시지를 편집기에 자동으로 채워준다.

Subject line (try to keep under 60 characters)

Multi-line description of commit,
feel free to be detailed.


\begin{verbatim}
[Ticket: X]
# Please enter the commit message for your changes. Lines starting
# with '#' will be ignored, and an empty message aborts the commit.
# On branch master
# Changes to be committed:
#   (use "git reset HEAD <file>..." to unstage)
#
# modified:   lib/test.rb
#
~
~
".git/COMMIT_EDITMSG" 14L, 297C

\end{verbatim}

소속 팀에 커밋 메시지 규칙이 있으면 그 규칙에 맞는 템플릿 파일을 만들고 시스템 설정에 설정해둔다. Git이 그 파일을 사용하도록 설정하면 규칙을 따르기가 쉬워진다.

core.pager
Git은 log 나 diff 같은 명령의 메시지를 출력할 때 페이지로 나누어 보여준다. 기본으로 사용하는 명령은 less 다. more 를 더 좋아하면 more 라고 설정한다. 페이지를 나누고 싶지 않으면 빈 문자열로 설정한다.

\$ git config --global core.pager ''
이 명령을 실행하면 Git은 길든지 짧든지 결과를 한 번에 다 보여 준다.

user.signingkey
이 설정은 내 작업에 서명하기 에서 설명했던 Annotated Tag를 만들 때 유용하다. 사용할 GPG 키를 설정해 둘 수 있다. 아래처럼 GPG 키를 설정하면 서명할 때 편리하다.

\$ git config --global user.signingkey <gpg-key-id>
git tag 명령을 실행할 때 키를 생략하고 서명할 수 있다.

\$ git tag -s <tag-name>
core.excludesfile
Git에서 git add 명령으로 추적할 파일에 포함하지 않을 파일은 .gitignore 에 해당 패턴을 적으면 된다고 파일 무시하기에서 설명했다.

한 저장소 안에서뿐 아니라 어디에서라도 Git에 포함하지 않을 파일을 설정할 수 있다. 
예를 들어 Mac을 쓰는 사람이라면 .DS\_Store 파일을 자주 보았을 것이다. 
Emacs나 Vim를 쓰다 보면 이름 끝에 \~, .swp 붙여둔 임시 파일도 있다.

.gitignore 파일처럼 무시할 파일을 설정할 수 있는데 ~/.gitignore\_global 파일 안에 아래 내용으로 입력해두고

*~
.*.swp
.DS\_Store
git config --global core.excludesfile ~/.gitignore\_global 명령으로 설정을 추가하면 더는 위와 같은 파일이 포함되지 않을 것이다.

help.autocorrect
명령어를 잘못 입력하면 Git은 메시지를 아래와 같이 보여 준다.

\$ git chekcout master
git: 'chekcout' is not a git command. See 'git --help'.

Did you mean this?
    checkout
Git은 어떤 명령을 입력하려고 했을지 추측해서 보여주긴 하지만 직접 실행하진 않는다. 그러나 help.autocorrect 를 1로 설정하면 명령어를 잘못 입력해도 Git이 자동으로 해당 명령어를 찾아서 실행해준다.

\$ git chekcout master
WARNING: You called a Git command named 'chekcout', which does not exist.
Continuing under the assumption that you meant 'checkout'
in 0.1 seconds automatically...
여기서 재밌는 것은 “0.1 seconds” 이다. 사실 help.autocorrect 설정에 사용하는 값은 1/10 초 단위의 숫자를 나타낸다. 만약 50이라는 값으로 설정한다면 자동으로 고친 명령을 실행할 때 Git은 5초간 명령을 실행하지 않고 기다려줄 수 있다.

\subsection{컬러 터미널}

\paragraph{}
사람이 쉽게 인식할 수 있도록 터미널에 결과를 컬러로 출력할 수 있다. 
터미널 컬러와 관련된 옵션은 매우 다양하기 때문에 꼼꼼하게 설정할 수 있다.

color.ui

\paragraph{기능 끄기}
Git은 기본적으로 터미널에 출력하는 결과물을 알아서 색칠하지만, 이 색칠하는 기능을 끄고 싶다면 한 가지 설정만 해 두면 된다. 
아래와 같은 명령을 실행하면 더는 색칠된 결과물을 내지 않는다.

\$ git config --global color.ui false

\paragraph{auto}
컬러 설정의 기본 값은 auto 로 터미널에 출력할 때는 색칠하지만, 결과가 리다이렉션되거나 파일로 출력되면 색칠하지 않는다.

\paragraph{always}
always 로 설정하면 터미널이든 다른 출력이든 상관없이 색칠하여 내보낸다. 
대개 이 값을 설정해서 사용하지 않는다. 

\paragraph{--color}
--color 옵션을 사용하면 강제로 결과를 색칠해서 내도록 할 수 있기 때문이다. 대부분은 기본 값이 낫다.

\paragraph{}
color.* \\
Git은 좀 더 꼼꼼하게 컬러를 설정하는 방법을 제공한다. 아래와 같은 설정들이 있다. 모두 true, false, always 중 하나를 고를 수 있다.


			\begin{itemize}[	topsep=0.0em,itemsep=0.0em,
							leftmargin=4em, labelsep=3em ]
			\item	color.branch 
			\item	color.diff 
			\item	color.interactive
			\item	color.status 
			\end{itemize}	


또한, 각 옵션의 컬러를 직접 지정할 수도 있다. 
아래처럼 설정하면 diff 명령에서 meta 정보의 포그라운드는 blue, 백그라운드는 black, 테스트는 bold로 바뀐다.

\$ git config --global color.diff.meta "blue black bold" \\

\paragraph{컬러}

컬러는 normal, black, red, green, yellow, blue, magenta, cyan, white 중에서 고를 수 있다. 


			\begin{itemize}	[	
								topsep=0.0em,
								itemsep=0.0em,
								leftmargin=4em, 
								labelsep=3em 
								]
			\item	normal, 
			\item	black, 
			\item	red, 
			\item	green, 
			\item	yellow, 
			\item	blue, 
			\item	magenta, 
			\item	cyan, 
			\item	white
			\end{itemize}	


\paragraph 	{텍스트 속성}

텍스트 속성은 bold, dim, ul (underline), blink, reverse 중에서 고를 수 있다.

			\begin{itemize}	[	
								topsep=0.0em,
								itemsep=0.0em,
								leftmargin=4em, 
								labelsep=3em 
								]
			\item	bold, 
			\item	dim, 
			\item	ul (underline), 
			\item	blink, 
			\item	reverse 
			\end{itemize}	



\subsection{다른 Merge, Diff 도구 사용하기}


Git에 들어 있는 diff 도구 말고 다른 도구로 바꿀 수 있다. 화려한 GUI 도구로 바꿔서 좀 더 편리하게 충돌을 해결할 수 있다. 여기서는 Perforce의 Merge 도구인 P4Merge로 설정하는 것을 보여준다. P4Merge는 무료인데다 꽤 괜찮다.

P4Merge는 중요 플랫폼을 모두 지원하기 때문에 웬만한 환경이면 사용할 수 있다. 여기서는 Mac과 Linux 시스템에 설치하는 것을 보여준다. Windows에서 사용하려면 /usr/local/bin 경로만 Windows 경로로 바꿔준다.

먼저 https://www.perforce.com/product/components/perforce-visual-merge-and-diff-tools 에서 P4Merge를 내려받는다. 
그 후에 P4Merge 에 쓸 Wrapper 스크립트를 만든다. 
필자는 Mac 사용자라서 Mac 경로를 사용한다. 
어떤 시스템이든 p4merge 가 설치된 경로를 사용하면 된다. 
예제에서는 extMerge 라는 Merge 용 Wrapper 스크립트를 만들고 이 스크립트로 넘어오는 모든 아규먼트를 p4merge 프로그램으로 넘긴다.

\begin{verbatim}
	\$ cat /usr/local/bin/extMerge
%	#! / bin / sh
\end{verbatim}

/Applications/p4merge.app/Contents/MacOS/p4merge \$*
그리고 diff용 Wrapper도 만든다. 이 스크립트로 넘어오는 아규먼트는 총 7개지만 그 중 2개만 Merge Wrapper로 넘긴다. Git이 diff 프로그램에 넘겨주는 아규먼트는 아래와 같다.

path old-file old-hex old-mode new-file new-hex new-mode
이 중에서 old-file 과 new-file 만 사용하는 wrapper 스크립트를 만든다.

\begin{verbatim}
\$ cat /usr/local/bin/extDiff
#!/bin/sh
[ \$# -eq 7 ] && /usr/local/bin/extMerge "\$2" "\$5"
\end{verbatim}

이 두 스크립트에 실행 권한을 부여한다.

\$ sudo chmod +x /usr/local/bin/extMerge
\$ sudo chmod +x /usr/local/bin/extDiff
Git config 파일에 이 스크립트를 모두 추가한다. 설정해야 하는 옵션이 좀 많다. merge.tool 로 무슨 Merge 도구를 사용할지, mergetool.*.cmd 로 실제로 어떻게 명령어를 실행할지, mergetool.trustExitCode 로 Merge 도구가 반환하는 exit 코드가 Merge의 성공 여부를 나타내는지, diff.external 은 diff 할 때 실행할 명령어가 무엇인지를 설정할 때 사용한다. 모두 git config 명령으로 설정한다.

\$ git config --global merge.tool extMerge
\$ git config --global mergetool.extMerge.cmd \
  'extMerge "\$BASE" "\$LOCAL" "\$REMOTE" "\$MERGED"'
\$ git config --global mergetool.extMerge.trustExitCode false
\$ git config --global diff.external extDiff
~/.gitconfig/ 파일을 직접 편집해도 된다.

[merge]
  tool = extMerge
[mergetool "extMerge"]
  cmd = extMerge "\$BASE" "\$LOCAL" "\$REMOTE" "\$MERGED"
  trustExitCode = false
[diff]
  external = extDiff
설정을 완료하고 나서 아래와 같이 diff 명령어를 실행한다.

\begin{verbatim}
\$ git diff 32d1776b1^ 32d1776b1
\end{verbatim}

diff 결과가 터미널에 출력되는 대신 P4Merge가 실행되어 아래처럼 Diff 결과를 보여준다.

P4Merge.
Figure 143. P4Merge.
브랜치를 Merge 할 때 충돌이 나면 git mergetool 명령을 실행한다. 이 명령을 실행하면 GUI 도구로 충돌을 해결할 수 있도록 P4Merge를 실행해준다.

Wrapper를 만들어 설정해두면 다른 Diff, Merge 도구로 바꾸기도 쉽다. 예를 들어, KDiff3를 사용하도록 extDiff 와 extMerge 스크립트를 수정한다.

\$ cat /usr/local/bin/extMerge
%#!/bin/sh
/Applications/kdiff3.app/Contents/MacOS/kdiff3 \$*
이제부터 Git은 Diff 결과를 보여주거나 충돌을 해결할 때 KDiff3 도구를 사용한다.

어떤 Merge 도구는 Git에 미리 설정이 들어 있다. 그래서 추가로 스크립트를 작성하거나 하는 설정 없이 사용할 수 있는 것도 있다. 아래와 같은 명령으로 확인해볼 수 있다.

\$ git mergetool --tool-help
'git mergetool --tool=<tool>' may be set to one of the following:
        emerge
        gvimdiff
        gvimdiff2
        opendiff
        p4merge
        vimdiff
        vimdiff2

The following tools are valid, but not currently available:
        araxis
        bc3
        codecompare
        deltawalker
        diffmerge
        diffuse
        ecmerge
        kdiff3
        meld
        tkdiff
        tortoisemerge
        xxdiff

Some of the tools listed above only work in a windowed
environment. If run in a terminal-only session, they will fail.
Diff 도구로는 다른 것을 사용하지만, Merge 도구로는 KDiff3를 사용하고 싶은 경우에는 kdiff3 명령을 실행경로로 넣고 아래와 같이 설정하기만 하면 된다.

\$ git config --global merge.tool kdiff3
extMerge 와 extDiff 파일을 만들지 않고 이렇게 Merge 도구만 `kdiff3`로 설정하고 Diff 도구는 Git에 원래 들어 있는 것을 사용할 수 있다.

Formatting and Whitespace
협업할 때 겪는 소스 포맷(Formatting)과 공백 문제는 미묘하고 난해하다. 동료 사이에 사용하는 플랫폼이 다를 때는 특히 더 심하다. 다른 사람이 보내온 Patch는 공백 문자 패턴이 미묘하게 다를 확률이 높다. 편집기가 몰래 공백문자를 추가해 버릴 수도 있고 크로스-플랫폼 프로젝트에서 Windows 개발자가 라인 끝에 CR(Carriage-Return) 문자를 추가해 버렸을 수도 있다. Git에는 이 이슈를 돕는 몇 가지 설정이 있다.

core.autocrlf
Windows에서 개발하는 동료와 함께 일하면 라인 바꿈(New Line) 문자에 문제가 생긴다. Windows는 라인 바꿈 문자로 CR(Carriage-Return)과 LF(Line Feed) 문자를 둘 다 사용하지만, Mac과 Linux는 LF 문자만 사용한다. 아무것도 아닌 것 같지만, 크로스 플랫폼 프로젝트에서는 꽤 성가신 문제다. Windows에서 사용하는 많은 편집기가 자동으로 LF 스타일의 라인 바꿈 스타일을 CRLF로 바꾸거나 Enter 키를 입력하면 CRLF 스타일을 사용하기 때문이다.

Git은 커밋할 때 자동으로 CRLF를 LF로 변환해주고 반대로 Checkout 할 때 LF를 CRLF로 변환해 주는 기능이 있다. core.autocrlf 설정으로 이 기능을 켤 수 있다. Windows에서 이 값을 true로 설정하면 Checkout 할 때 LF 문자가 CRLF 문자로 변환된다.

\$ git config --global core.autocrlf true
라인 바꿈 문자로 LF를 사용하는 Linux와 Mac에서는 Checkout 할 때 Git이 LF를 CRLF로 변환할 필요가 없다. 게다가 우연히 CRLF가 들어간 파일이 저장소에 들어 있어도 Git이 알아서 고쳐주면 좋을 것이다. core.autocrlf 값을 input으로 설정하면 커밋할 때만 CRLF를 LF로 변환한다.

\$ git config --global core.autocrlf input
이 설정을 이용하면 Windows에서는 CRLF를 사용하고 Mac, Linux, 저장소에서는 LF를 사용할 수 있다.

Windows 플랫폼에서만 개발하면 이 기능이 필요 없다. 이 옵션을 false 라고 설정하면 이 기능이 꺼지고 CR 문자도 저장소에도 저장된다.

\$ git config --global core.autocrlf false
core.whitespace
Git에는 공백 문자를 다루는 방법으로 네 가지가 미리 정의돼 있다. 두 가지는 기본적으로 켜져 있지만 끌 수 있고 나머지 두 가지는 꺼져 있지만 켤 수 있다.

먼저 기본적으로 켜져 있는 것을 살펴보자. blank-at-eol 는 각 라인 끝에 공백이 있는지 찾고, blank-at-eof 는 파일 끝에 추가한 빈 라인이 있는지 찾고, space-before-tab 은 모든 라인에서 처음에 tab보다 공백이 먼저 나오는지 찾는다.

기본적으로 꺼져 있는 나머지 세 개는 indent-with-non-tab 과 tab-in-indent 과 cr-at-eol 이다. intent-with-non-tab 은 tab이 아니라 공백으로(tabwidth 설정에 영향받음) 시작하는 라인이 있는지 찾고 cr-at-eol 은 라인 끝에 CR 문자가 있어도 괜찮다고 Git에 알리는 것이다.

core.whitespace 옵션으로 이 네 가지 방법을 켜고 끌 수 있다. 설정에서 해당 옵션을 빼버리거나 이름이 - 로 시작하면 기능이 꺼진다. 예를 들어, 다른 건 다 켜고 space-before-tab 옵션만 끄려면 아래와 같이 설정한다. trailing-space 옵션은 blank-at-eol 옵션과 blank-at-eof 옵션을 의미한다.

\$ git config --global core.whitespace \
    trailing-space,-space-before-tab,indent-with-non-tab,tab-in-indent,cr-at-eol
또는 각 부분에 대해서 설정을 할 수도 있다.

\$ git config --global core.whitespace \
    -space-before-tab,indent-with-non-tab,tab-in-indent,cr-at-eol
git diff 명령을 실행하면 Git은 이 설정에 따라 검사해서 컬러로 표시해준다. 그래서 좀 더 쉽게 검토해서 커밋할 수 있다. git apply 명령으로 Patch를 적용할 때도 이 설정을 이용할 수 있다. 아래처럼 명령어를 실행하면 해당 Patch가 공백문자 정책에 들어맞는지 확인할 수 있다.

\$ git apply --whitespace=warn <patch>
아니면 Git이 자동으로 고치도록 할 수 있다.

\$ git apply --whitespace=fix <patch>
이 옵션은 git rebase 명령에서도 사용할 수 있다. 공백 문제가 있는 커밋을 Upstream에 Push 하기 전에 --whitespace=fix 옵션을 주고 Rebase 하면 Git은 다시 Patch를 적용하면서 공백을 설정한 대로 고친다.



\subsection{서버 설정}
서버 설정은 많지 않지만, 꼭 짚고 넘어가야 하는 것이 몇 개 있다.

receive.fsckObjects
Git은 Push 할 때마다 각 개체가 SHA-1 체크섬에 맞는지 잘못된 개체가 가리키고 있는지 검사하게 할 수 있다. 기본적으로 이 기능이 동작하지 않게 설정이 되어 있는데 개체를 점검하데 상당히 시간이 걸리기 때문에 Push 하는 시간이 늘어난다. 얼마나 늘어나는지는 저장소 크기와 Push 하는 양에 달렸다. 

\paragraph{}
receive.fsckOBjects 값을 true로 설정하면 Git이 Push 할 때마다 검증한다.

\$ git config --system receive.fsckObjects true
이렇게 설정하면 Push 할 때마다 검증하기 때문에 클라이언트는 잘못된 데이터를 Push 하지 못한다.

receive.denyNonFastForwards
이미 Push 한 커밋을 Rebase 해서 다시 Push 하지 못하게 할 수 있다. 브랜치를 Push 할 때 해당 리모트 브랜치가 가리키는 커밋이 Push 하려는 브랜치에 없을 때 Push 하지 못하게 할 수 있다. 보통은 이런 정책이 좋고 git push 명령에 -f 옵션을 주면 강제로 Push 할 수 있다.

receive.denyNonFastForwards 옵션을 켜면 Fast-forward로 Push 할 수 없는 브랜치는 아예 Push 하지 못한다.

\$ git config --system receive.denyNonFastForwards true
사용자마다 다른 정책을 적용하고 싶으면 서버 훅을 사용해야 한다. 서버의 receive 훅으로 할 수 있고 이 훅도 이 장에서 설명한다.

receive.denyDeletes
receive.denyNonFastForwards 와 비슷한 정책으로 receive.denyDeletes 라는 것이 있다. 이 설정을 켜면 브랜치를 삭제하는 Push가 거절된다.

\$ git config --system receive.denyDeletes true
이제 브랜치나 Tag를 삭제하는 Push는 거절된다. 아무도 삭제할 수 없다. 리모트 브랜치를 삭제하려면 직접 손으로 server의 ref 파일을 삭제해야 한다. 그리고 사용자마다 다른 정책을 적용시키는 ACL을 만드는 방법도 있다. 이 방법은 정책 구현하기 에서 다룬다.



%	================================================================== Part			git ignore
	\addtocontents{toc}{\protect\newpage}
	\chapter 	{git ignore}

	\noptcrule
	\minitoc
				

% ----------------------------------------------------------------------------- 	git ignore 란
%										
% -----------------------------------------------------------------------------										
	\section{git ignore 란}


gitignore 파일이란 git버전관리에서 제외할 파일 목록을 지정하는 파일이다.


% -----------------------------------------------------------------------------										
	\section{git ignore 문법}

% -----------------------------------------------------------------------------										
	\section{git ignore 설정하기}

% -----------------------------------------------------------------------------										
	\section{git ignore 적용하기}

% -----------------------------------------------------------------------------										
	\section{git ignore 설정 확인}

% -----------------------------------------------------------------------------										
	\section{git ignore 설정해제}

% -----------------------------------------------------------------------------										
	\section{git ignore 추가하기}

% -----------------------------------------------------------------------------										
	\section{github에 이미 올라가 있는 파일 삭제하고 gitignore 적용하기}

% -----------------------------------------------------------------------------										
	\section{github에 잘못 올라간 파일 삭제하기}










%	================================================================== Part			브랜치
	\addtocontents{toc}{\protect\newpage}
	\part 	{브랜치}

	\noptcrule
	\parttoc	
	\minitoc
				


% ----------------------------------------------------------------------------- 	브랜치
%										
% -----------------------------------------------------------------------------										
	\section{브랜치}

			\begin{itemize}	[	
							topsep=0.0em,
							itemsep=0.0em,
							leftmargin=6em, 
							labelsep=3em 
							]
				\item 	branch 목록 보기
				\item 	옵션: origin
				\item	원격 branch 목록 보기
				\item	branch 이동: checkout
				\item	브랜치 병합: merge
				\item	branch 삭제, 목록 업데이트
			\end{itemize}


master branch란 모든 repository의 기본 혹은 메인이라고 보면 된다. 
일반적으로 repo의 모든 것은 master branch를 중심으로 행해진다.

큰 프로젝트든 개인 프로젝트이든 최종 결과물은 master branch에 있기 마련이며, 
master branch로부터 파생된 다른 branch들로부터 수정 사항을 만든 다음 
master에 병합하는 과정을 거치게 된다.

여러 사람이 협업할 경우 각자 따로 브랜치를 쓰게 되며, 
각 브랜치에서는 새로운 기능을 개발하거나 버그 수정이 이루어진다. 
물론 완료되면 master branch에 병합되게 된다.

위의 설명이 정석적인 git repo의 운영방법이고, 
master branch에는 일반적으로 직접 수정을 가하지 않는다. 
따라서 별다른 일이 없다면 본 글에서부터는 master branch에 직접 commit을 날리지 않고, 
branch를 만든 다음 병합하는 과정을 거칠 것이다.

잠깐 브라우저를 켜서 브랜치 부분을 클릭해 보자.

\paragraph{branch 목록 보기}

목록을 보는 옵션은 여러 가지가 있다. git branch 혹은 git branch --list를 입력해 보자.

		\begin{tcolorbox}
 		git branch  --list
		\end{tcolorbox}

위 명령은 local repo에 저장되어 있는 branch들의 리스트를 보여 준다. 
다른 branch를 만들지 않았기 때문에 master 하나밖에 보이지 않을 것이다.

이제 여러분은 다음과 같은 형태의 트리를 갖고 있다.
그렇다. branch는 tree의 것이다.

조금 더 자세하게 그리기 위해, git log --oneline을 명령창에 입력한다. 
무엇을 하는 것인지 잊어버리지는 않았을 것이다.

		\begin{tcolorbox}
 		git branch  -- one line
		\end{tcolorbox}


\paragraph{원격 branch 목록 보기}
이제 dummy repo는 잊어버리고, 원래 하던 것으로 돌아오자.

local repo 말고 remote repo의 브랜치를 알고 싶다면 다음 중 하나를 입력한다.

		\begin{tcolorbox}
		git branch -r
		git branch –remote
		\end{tcolorbox}

local이랑 remote 전부 보고 싶으면 다음을 입력한다.

		\begin{tcolorbox}
		git branch -a
		git branch –all
		\end{tcolorbox}

		\begin{itemize}	[							leftmargin=6em]
		\item 	-r 옵션과 -a 옵션의 remote repo 표기가 조금 다르다.
		\item -a 옵션은 local repo와 remote repo를 구분하기 위해 ‘remotes/’를 remote repo 앞에 붙인다.
		\item -r 옵션은 remote repo만 보여주기 때문에 ‘remotes/’ 표시가 필요 없다.
		\end{itemize}

	\section{브랜치 만들기 }

		'issue1' 이라는 이름으로 새로운 브랜치를 작성합니다.\\
		브랜치는 branch 란 명령어로 만들 수 있습니다.

		\begin{tcolorbox}
		\$ git branch 〈branchname〉
		\end{tcolorbox}


'issue1' 이라는 이름으로 브랜치를 만들어 봅시다.

		\begin{tcolorbox}
		\$ git branch issue1
		\end{tcolorbox}

옵션을 지정하지 않고 branch 명령어를 실행하면 브랜치 목록 전체를 확인할 수 있습니다. 앞 부분에 * 이 붙어있는 것이 현재 선택된 브랜치입니다.
		\begin{tcolorbox}
		\$ git branch  \\
		  issue1  \\
		* master  
		\end{tcolorbox}

이 시점까지의 이력을 보면 다음과 같습니다.






	\section{브랜치 전환하기 }
	\section{브랜치 병합하기 }
	\section{브랜치 삭제하기 }
	\section{브랜치 동시에 여러 작업 }
	\section{브랜치 종류 }


Git Branch 종류 (5가지)
Gitflow Workflow에서는 항상 유지되는 메인 브랜치들(master, develop)과 
일정 기간 동안만 유지되는 보조 브랜치들(feature, release, hotfix)을 포함하여 총 5가지의 브랜치를 사용한다.

			\begin{itemize}	[	
							topsep=0.0em,
							itemsep=0.0em,
							leftmargin=6em, 
							labelsep=3em 
							]
				\item	master,
				\item	develop
				\item	feature, 
				\item	release, 
				\item	hotfix
			\end{itemize}
아래는 Gitflow Workflow 방법에서 사용하는 브랜치의 흐름이다.


	\subsection{1. Master Branch}
		제품으로 출시될 수 있는 브랜치
		배포(Release) 이력을 관리하기 위해 사용. 즉, 배포 가능한 상태만을 관리한다.
		https://gmlwjd9405.github.io/2018/05/11/types-of-git-branch.html

	\subsection{2. Develop Branch}
다음 출시 버전을 개발하는 브랜치
기능 개발을 위한 브랜치들을 병합하기 위해 사용. 즉, 모든 기능이 추가되고 버그가 수정되어 배포 가능한 안정적인 상태라면 develop 브랜치를 ‘master’ 브랜치에 병합(merge)한다.
평소에는 이 브랜치를 기반으로 개발을 진행한다.
https://gmlwjd9405.github.io/2018/05/11/types-of-git-branch.html


	\subsection{3. Feature branch}

		\paragraph{기능을 개발하는 브랜치}
		feature 브랜치는 새로운 기능 개발 및 버그 수정이 필요할 때마다 ‘develop’ 브랜치로부터 분기한다. 
		feature 브랜치에서의 작업은 기본적으로 공유할 필요가 없기 때문에, 자신의 로컬 저장소에서 관리한다.
		개발이 완료되면 ‘develop’ 브랜치로 병합(merge)하여 다른 사람들과 공유한다.
		
		\paragraph{}

		\begin{itemize}
			\item		‘develop’ 브랜치에서 새로운 기능에 대한 feature 브랜치를 분기한다.
			\item		새로운 기능에 대한 작업 수행한다.
			\item		작업이 끝나면 ‘develop’ 브랜치로 병합(merge)한다.
			\item		더 이상 필요하지 않은 feature 브랜치는 삭제한다.
			\item		새로운 기능에 대한 ‘feature’ 브랜치를 중앙 원격 저장소에 올린다.(push)
		\end{itemize}

		\paragraph{feature 브랜치 이름 정하기}

		\begin{itemize}
			\item		master, develop, release-(RB\_), or hotfix- 제외
			\item		(feature/기능요약) 형식을 추천 EX) feature/login
		\end{itemize}


		\paragraph{feature 브랜치 생성 및 종료 과정}
		.\\	


		\begin{tcolorbox}
			// feature 브랜치(feature/login)를 'develop' 브랜치('master' 브랜치에서 따는 것이 아니다!)에서 분기\\
			\$ git checkout -b feature/login develop\\
			/* ~ 새로운 기능에 대한 작업 수행 ~ */\\

		/* feature 브랜치에서 모든 작업이 끝나면 */\\
		// 'develop' 브랜치로 이동한다.\\
		\$ git checkout develop \\
		// 'develop' 브랜치에 feature/login 브랜치 내용을 병합(merge)한다. \\
		 --no-ff 옵션: 아래에 추가 설명\\
		\$ git merge --no-ff feature/login\\
		// -d 옵션: feature/login에 해당하는 브랜치를 삭제한다.\\
		\$ git branch -d feature/login	\\
		// 'develop' 브랜치를 원격 중앙 저장소에 올린다.	\\
		\$ git push origin develop 
		\end{tcolorbox}

		\paragraph{--no-ff 옵션}
		새로운 커밋 객체를 만들어 ‘develop’ 브랜치에 merge 한다.
		이것은 ‘feature’ 브랜치에 존재하는 커밋 이력을 모두 합쳐서 
		하나의 새로운 커밋 객체를 만들어 ‘develop’ 브랜치로 병합(merge)하는 것이다.




	\subsection{4. Release Branch}

			\paragraph{이번 출시 버전을 준비하는 브랜치}
			배포를 위한 전용 브랜치를 사용함으로써 한 팀이 해당 배포를 준비하는 동안 다른 팀은 다음 배포를 위한 기능 개발을 계속할 수 있다. 
			즉, 딱딱 끊어지는 개발 단계를 정의하기에 아주 좋다.
			예를 들어, ‘이번 주에 버전 1.3 배포를 목표로 한다!’라고 팀 구성원들과 쉽게 소통하고 합의할 수 있다는 말이다.
			
		\begin{itemize}
\item		‘develop’ 브랜치에서 배포할 수 있는 수준의 기능이 모이면 또는 정해진 배포 일정이 되면, release 브랜치를 분기한다.
			\begin{itemize}
			\item		release 브랜치를 만드는 순간부터 배포 사이클이 시작된다.
			\item		release 브랜치에서는 배포를 위한 최종적인 버그 수정, 문서 추가 등 릴리스와 직접적으로 관련된 작업을 수행한다.
			\item		직접적으로 관련된 작업들을 제외하고는 release 브랜치에 새로운 기능을 추가로 병합(merge)하지 않는다.
			\end{itemize}
\item		‘release’ 브랜치에서 배포 가능한 상태가 되면(배포 준비가 완료되면),
			\begin{itemize}
			\item		배포 가능한 상태: 새로운 기능을 포함한 상태로 모든 기능이 정상적으로 동작 하는 상태
			\item		‘master’ 브랜치에 병합한다. (이때, 병합한 커밋에 Release 버전 태그를 부여!)
			\item		배포를 준비하는 동안 release 브랜치가 변경되었을 수 있으므로 배포 완료 후 ‘develop’ 브랜치에도 병합한다.
			\end{itemize}
			\end{itemize}


			이때, 다음 번 배포(Release)를 위한 개발 작업은 ‘develop’ 브랜치에서 계속 진행해 나간다.
			
			\paragraph{}release 브랜치 이름 정하기

			\begin{itemize}
				\item release-RB\_* 또는 release-* 또는 release/* 처럼 이름 짓는 것이 일반적인 관례
				\item [release-* ] 형식을 추천 EX) release-1.2
			\end{itemize}
			
			
			\paragraph{release 브랜치 생성 및 종료 과정}
.\\
			
		\begin{tcolorbox}
			// release 브랜치(release-1.2)를 'develop' 브랜치('master' 브랜치에서 따는 것이 아니다!)에서 분기
			\$ git checkout -b release-1.2 develop \\
			/* ~ 배포 사이클이 시작 ~ */ \\
			/* release 브랜치에서 배포 가능한 상태가 되면 */  \\
			// 'master' 브랜치로 이동한다.  \\
			\$ git checkout master  \\
			// 'master' 브랜치에 release-1.2 브랜치 내용을 병합(merge)한다. \\
			\# --no-ff 옵션: 위의 추가 설명 참고 \\
			\$ git merge --no-ff release-1.2 \\
			// 병합한 커밋에 Release 버전 태그를 부여한다. \\
			\$ git tag -a 1.2 \\
			/* 'release' 브랜치의 변경 사항을 'develop' 브랜치에도 적용 */  \\
			// 'develop' 브랜치로 이동한다.  \\
			\$ git checkout develop  \\
			// 'develop' 브랜치에 release-1.2 브랜치 내용을 병합(merge)한다.  \\
			\$ git merge --no-ff release-1.2  \\
			// -d 옵션: release-1.2에 해당하는 브랜치를 삭제한다.  \\
			\$ git branch -d release-1.2  \\
		\end{tcolorbox}

			

	\section{브랜치}



			\paragraph{출시 버전에서 발생한 버그를 수정 하는 브랜치} 
			배포한 버전에 긴급하게 수정을 해야 할 필요가 있을 경우, ‘master’ 브랜치에서 분기하는 브랜치이다. 
			‘develop’ 브랜치에서 문제가 되는 부분을 수정하여 배포 가능한 버전을 만들기에는 시간도 많이 소요되고 안정성을 보장하기도 어려우므로 
			바로 배포가 가능한 ‘master’ 브랜치에서 직접 브랜치를 만들어 필요한 부분만을 수정한 후 다시 ‘master’브랜치에 병합하여 이를 배포해야 하는 것이다.
			
			\begin{itemize}
			\item 배포한 버전에 긴급하게 수정을 해야 할 필요가 있을 경우,
				\begin{itemize}
					\item ‘master’ 브랜치에서 hotfix 브랜치를 분기한다. (‘hotfix’ 브랜치만 master에서 바로 딸 수 있다.)
				\end{itemize}
			\item 문제가 되는 부분만을 빠르게 수정한다.
				\begin{itemize}
					\item 다시 ‘master’ 브랜치에 병합(merge)하여 이를 안정적으로 다시 배포한다.
					\item 새로운 버전 이름으로 태그를 매긴다.
					\item hotfix 브랜치에서의 변경 사항은 ‘develop’ 브랜치에도 병합(merge)한다.
				\end{itemize}
			\end{itemize}
			
			
			버그 수정만을 위한 ‘hotfix’ 브랜치를 따로 만들었기 때문에, 다음 배포를 위해 개발하던 작업 내용에 전혀 영향을 주지 않는다. ‘hotfix’ 브랜치는 master 브랜치를 부모로 하는 임시 브랜치라고 생각하면 된다.
			
				\begin{itemize}
					\item hotfix 브랜치 이름 정하기 \\
							\[hotfix-* \] 형식을 추천 EX) hotfix-1.2.1
					\item hotfix 브랜치 생성 및 종료 과정
				\end{itemize}



		\begin{tcolorbox}
		\begin{quote}
			// release 브랜치(hotfix-1.2.1)를 'master' 브랜치(유일!)에서 분기\\
			\$ git checkout -b hotfix-1.2.1 master \\
			/* ~ 문제가 되는 부분만을 빠르게 수정 ~ */ \\
			/* 필요한 부분을 수정한 후 */ \\
			// 'master' 브랜치로 이동한다.  \\
			\$ git checkout master \\
			// 'master' 브랜치에 hotfix-1.2.1 브랜치 내용을 병합(merge)한다.  \\
			\$ git merge --no-ff hotfix-1.2.1 \\
			// 병합한 커밋에 새로운 버전 이름으로 태그를 부여한다. \\
			\$ git tag -a 1.2.1  \\
			/* 'hotfix' 브랜치의 변경 사항을 'develop' 브랜치에도 적용 */  \\
			// 'develop' 브랜치로 이동한다.  \\
			\$ git checkout develop  \\
			// 'develop' 브랜치에 hotfix-1.2.1 브랜치 내용을 병합(merge)한다.  \\
			\$ git merge --no-ff hotfix-1.2.1  \\
		\end{quote}
		\end{tcolorbox}



%	================================================================== Part			마크 다운 작성		마크다운
	\addtocontents{toc}{\protect\newpage}
	\part 	{마크 다운 작성 }
	\noptcrule
	\parttoc				

			
			% ----------------------------------------------------------------------------- 	마크 다운 작성
			%										
			% -----------------------------------------------------------------------------										
				\section{마크 다운 작성}
			
			
				\section{마크다운의 장점}
				문법이 쉽다.
				관리가 쉽다.
				지원 가능한 플랫폼과 프로그램이 다양하다.
			
				\section{마크다운의 단점}
				표준이 없어 사용자마다 문법이 상이할 수 있다.
				모든 HTML 마크업을 대신하지 못한다.
			
				\section{마크다운의 사용}
			
				메모장부터 전용 에디터까지 많은 곳에서 활용할 수 있습니다.
				문법이 쉽기 때문에 꼭 전용 에디터를 사용할 필요는 없습니다만, 마크다운 코드의 하이라이트 효과를 원한다면 전용 에디터가 좋은 선택이 될 것 같네요.
				저는 평소 Atom을 사용하고 있습니다.
				혹은 마크다운 문법을 지원하는 모든 곳에서 사용할 수 있으며, 일반 블로그나 워드프레스 외 Slack이나 Trello 같은 서비스에서 메세지를 작성하듯 사용할 수도 있습니다.
				화면에 표현되는 스타일(CSS)은 설정에 따라 달라집니다.
				HTML과 마찬가지로 눈에 보이는 스타일은 무시하고 각 문법의 의미로 사용하세요.
				

			% -----------------------------------------------------------------------------										
				\chapter 	{마크다운 문법(syntax)}
				\minitoc 
			
				\section{제목(Header)}
				〈h1〉부터 〈h6〉까지 제목을 표현할 수 있습니다.
				
				\# 제목 1
				\#\# 제목 2
				\#\#\# 제목 3
				\#\#\#\# 제목 4
				\#\#\#\#\# 제목 5
				\#\#\#\#\#\# 제목 6
				제목1(h1)과 제목2(h2)는 다음과 같이 표현할 수 있습니다.
				
				제목 1
				======
				
				제목 2
				------
						
				\section{강조(Emphasis)}
				
				각각 〈em〉, 〈strong〉, 〈del〉 태그로 변환됩니다.
				
				밑줄을 입력하고 싶다면 〈u〉〈/u〉 태그를 사용하세요.
				
				이텔릭체는 *별표(asterisks)* 혹은 \_언더바(underscore)\_를 사용하세요.
				두껍게는 **별표(asterisks)** 혹은 \_\_언더바(underscore)\_\_를 사용하세요.
				**\_이텔릭체\_와 두껍게**를 같이 사용할 수 있습니다.
				취소선은 ~~물결표시(tilde)~~를 사용하세요.
				〈u〉밑줄〈/u〉은 `〈u〉〈/u〉`를 사용하세요.
				이텔릭체는 별표(asterisks) 혹은 언더바(underscore)를 사용하세요.
				두껍게는 별표(asterisks) 혹은 언더바(underscore)를 사용하세요.
				이텔릭체와 두껍게를 같이 사용할 수 있습니다.
				취소선은 물결표시(tilde)를 사용하세요.
				밑줄은 〈u〉〈/u〉를 사용하세요.
			
				\section{목록(List)}
					
					〈ol〉, 〈ul〉 목록 태그로 변환됩니다.
					
					1. 순서가 필요한 목록
					1. 순서가 필요한 목록
					  - 순서가 필요하지 않은 목록(서브) 
					  - 순서가 필요하지 않은 목록(서브) 
					1. 순서가 필요한 목록
					  1. 순서가 필요한 목록(서브)
					  1. 순서가 필요한 목록(서브)
					1. 순서가 필요한 목록
					
					- 순서가 필요하지 않은 목록에 사용 가능한 기호
					  - 대쉬(hyphen)
					  * 별표(asterisks)
					  + 더하기(plus sign)
					순서가 필요한 목록
					순서가 필요한 목록
					순서가 필요하지 않은 목록(서브)
					순서가 필요하지 않은 목록(서브)
					순서가 필요한 목록
					순서가 필요한 목록(서브)
					순서가 필요한 목록(서브)
					순서가 필요한 목록
					순서가 필요하지 않은 목록에 사용 가능한 기호
					대쉬(hyphen)
					별표(asterisks)
					더하기(plus sign)
			
				\section{링크(Links)}
					
					〈a〉로 변환됩니다.
					
%					[GOOGLE](https://google.com) \\
%					[NAVER](https://naver.com "링크 설명(title)을 작성하세요.") \\
%					[상대적 참조](../users/login) \\
%					[Dribbble][Dribbble link] \\%
%					[GitHub][1] \\
					
%					문서 안에서 [참조 링크]를 그대로 사용할 수도 있습니다.
%					
%					다음과 같이 문서 내 일반 URL이나 꺾쇠 괄호(`〈 〉`, Angle Brackets)안의 URL은 자동으로 링크를 사용합니다.
%					구글 홈페이지: https://google.com
%					네이버 홈페이지: 〈https://naver.com〉
%					
%					[Dribbble link]: https://dribbble.com
%					[1]: https://github.com
%					[참조 링크]: https://naver.com "네이버로 이동합니다!"
%					GOOGLE
					
%					NAVER
%					
%					상대적 참조
%					
%					Dribbble
%					
%					GitHub
%					
%					문서 안에서 참조 링크를 그대로 사용할 수도 있습니다.
%					
%					다음과 같이 문서 내 일반 URL이나 꺾쇠 괄호(〈 〉, Angle Brackets)안의 URL은 자동으로 링크를 사용합니다.
%					
%					구글 홈페이지: https://google.com
%					네이버 홈페이지: https://naver.com
			
				\section{이미지(Images)}
	
	〈img〉로 변환됩니다.
	링크과 비슷하지만 앞에 !가 붙습니다.
	
	![대체 텍스트(alternative text)를 입력하세요!](http://www.gstatic.com/webp/gallery/5.jpg "링크 설명(title)을 작성하세요.")
	
	![Kayak][logo]
	
	[logo]: http://www.gstatic.com/webp/gallery/2.jpg "To go kayaking."
	대체 텍스트(alternative text)를 입력하세요!
	
	Kayak
			
				\section{이미지에 링크}
%			%	마크다운 이미지 코드를 링크 코드로 묶어 줍니다.
%			%	
%			%	[![Vue](/images/vue.png)](https://kr.vuejs.org/)
%			%	Vue
%			
%				\section{코드(Code) 강조}
%			%	〈pre〉, 〈code〉로 변환됩니다.
%			%	숫자 1번 키 왼쪽에 있는 `(Grave)를 입력하세요
%			
%			\paragraph{인라인(inline) 코드 강조}
%			%	`background`혹은 `background-image` 속성으로 요소에 배경 이미지를 삽입할 수 있습니다.
%			%	background혹은 background-image 속성으로 요소에 배경 이미지를 삽입할 수 있습니다.
%			
%			\paragraph{블록(block) 코드 강조}
%			%	`를 3번 이상 입력하고 코드 종류도 적습니다.
%			%	
%			%	
%			%	```html
%			%	〈a href="https://www.google.co.kr/" target="_blank"〉GOOGLE〈/a〉
%			%	```
%			%	
%			%	```css
%			%	.list 〉 li {
%			%	  position: absolute;
%			%	  top: 40px;
%			%	}
%			%	```
%			%	
%			%	```javascript
%			%	function func() {
%			%	  var a = 'AAA';
%			%	  return a;
%			%	}
%			%	```
%			%	
%			%	```bash
%			%	$ vim ./~zshrc
%			%	```
%			%	
%			%	```python
%			%	s = "Python syntax highlighting"
%			%	print s
%			%	```
%			%	
%			%	```
%			%	No language indicated, so no syntax highlighting. 
%			%	But let's throw in a tag.
%			%	```
%			%	
%			%	〈a href="https://www.google.co.kr/" target="_blank"〉GOOGLE〈/a〉
%			%	.list 〉 li {
%			%	  position: absolute;
%			%	  top: 40px;
%			%	}
%			%	function func() {
%			%	  var a = 'AAA';
%			%	  return a;
%			%	}
%			%	$ vim ./~zshrc
%			%	s = "Python syntax highlighting"
%			%	print s
%			%	No language indicated, so no syntax highlighting. 
%			%	But let's throw in a 〈b〉tag〈/b〉.
%			
				\section{표(Table)}
				
					〈table〉 태그로 변환됩니다.
%					헤더 셀을 구분할 때 3개 이상의 -(hyphen/dash) 기호가 필요합니다.
%					헤더 셀을 구분하면서 :(Colons) 기호로 셀(열/칸) 안에 내용을 정렬할 수 있습니다.
%					가장 좌측과 가장 우측에 있는 |(vertical bar) 기호는 생략 가능합니다.
%					
%					| 값 | 의미 | 기본값 |
%					|---|:---:|---:|
%					| `static` | 유형(기준) 없음 / 배치 불가능 | `static` |
%					| `relative` | 요소 자신을 기준으로 배치 |  |
%					| `absolute` | 위치 상 부모(조상)요소를 기준으로 배치 |  |
%					| `fixed` | 브라우저 창을 기준으로 배치 |  |
%					
%					값 | 의미 | 기본값
%					---|:---:|---:
%					`static` | 유형(기준) 없음 / 배치 불가능 | `static`
%					`relative` | 요소 **자신**을 기준으로 배치 |
%					`absolute` | 위치 상 **_부모_(조상)요소**를 기준으로 배치 |
%					`fixed` | **브라우저 창**을 기준으로 배치 |
%					값	의미	기본값
%					static	유형(기준) 없음 / 배치 불가능	static
%					relative	요소 자신을 기준으로 배치	
%					absolute	위치 상 부모(조상)요소를 기준으로 배치	
%					fixed	브라우저 창을 기준으로 배치	
%					인용문(BlockQuote)
%					〈blockquote〉 태그로 변환됩니다.
				
				\section{인용문(blockQuote)}
			
					〉 남의 말이나 글에서 직접 또는 간접으로 따온 문장.
					〉\_(네이버 국어 사전)\_
					
					BREAK!
					
					〉 인용문을 작성하세요!
					〉〉 중첩된 인용문(nested blockquote)을 만들 수 있습니다.
					〉〉〉 중중첩된 인용문 1
					〉〉〉 중중첩된 인용문 2
					〉〉〉 중중첩된 인용문 3
					인용문(blockQuote)
					
					남의 말이나 글에서 직접 또는 간접으로 따온 문장.
					(네이버 국어 사전)
					
					BREAK!
					
					인용문을 작성하세요!
					
					중첩된 인용문(nested blockquote)을 만들 수 있습니다.
					
					중중첩된 인용문 1
					중중첩된 인용문 2
					중중첩된 인용문 3
					
				\section{원시 HTML(Raw HTML)}
					마크다운 문법이 아닌 원시 HTML 문법을 사용할 수 있습니다.
					
					〈u〉마크다운에서 지원하지 않는 기능〈/u〉을 사용할 때 유용하며 대부분 잘 동작합니다.
					
					〈img width="150" src="http://www.gstatic.com/webp/gallery/4.jpg" alt="Prunus" title="A Wild Cherry (Prunus avium) in flower"〉
					
					![Prunus](http://www.gstatic.com/webp/gallery/4.jpg)
					마크다운에서 지원하지 않는 기능을 사용할 때 유용하며 대부분 잘 동작합니다.
					
					Prunus
					
					Prunus
				
				\section{수평선(Horizontal Rule)}
				
					각 기호를 3개 이상 입력하세요.
					
					\-\-\-
					(Hyphens)
					
					***
					(Asterisks)
					
					\_\_\_
					(Underscores)
					(Hyphens)
					
					(Asterisks)
					
					(Underscores)
				
				\section{줄바꿈(Line Breaks)}
					
					동해물과 백두산이 마르고 닳도록 
					하느님이 보우하사 우리나라 만세   〈!--띄어쓰기 2번--〉
					무궁화 삼천리 화려 강산〈br〉
					대한 사람 대한으로 길이 보전하세
					동해물과 백두산이 마르고 닳도록
					하느님이 보우하사 우리나라 만세
					무궁화 삼천리 화려 강산
					대한 사람 대한으로 길이 보전하세
					
					일반 줄비꿈이 동작하지 않는 환경(설정 및 버전에 따라)의 경우, ‘2번의 띄어쓰기’나 〈br〉를 활용할 수 있습니다.
			

%	================================================================== Part		텍스트 편집기
%	\addtocontents{toc}{\protect\newpage}
	\part	{텍스트 편집기}
	\noptcrule
	\parttoc				




%	================================================================== Chapter		VIM
%	\addtocontents{toc}{\protect\newpage}
	\chapter	{VIM}
	\noptcrule
	\minitoc


% ----------------------------------------------------------------------------- 	VIM 개요
%										
% -----------------------------------------------------------------------------										
	\section{VIM 개요}


% ----------------------------------------------------------------------------- 	VIM 설치
%										
% -----------------------------------------------------------------------------										
	\section{VIM 설치}



%	================================================================== Chapter		Atom
%	\addtocontents{toc}{\protect\newpage}
	\chapter	{Atom}
	\noptcrule
	\minitoc


% ----------------------------------------------------------------------------- 	Atom 개요
%										
% -----------------------------------------------------------------------------									
	\section{Atom 개요}


% ----------------------------------------------------------------------------- 	Atom 설치
%										
% -----------------------------------------------------------------------------									
	\section{Atom 설치}


%	================================================================== Chapter		VS Code 비주얼스튜디오코드
%	\addtocontents{toc}{\protect\newpage}
	\chapter	{VS Code}
	\noptcrule
	\minitoc


% ----------------------------------------------------------------------------- 	VS Code 개요
%										 
% -----------------------------------------------------------------------------									
	\section 	{VS Code 개요}


% ----------------------------------------------------------------------------- 	VS Code 설치
%										
% -----------------------------------------------------------------------------									
	\section{VS Code 설치}



% ----------------------------------------------------------------------------- 	VS Code 화면 구성
%										
% -----------------------------------------------------------------------------									
	\section{VS Code 화면 구성}



% ----------------------------------------------------------------------------- 	VS Code 색 테마 바꾸기
%										
% -----------------------------------------------------------------------------									
	\section{VS Code 색 테마 바꾸기}



%	================================================================== Part		Github Desktop 	깃허브데스크탑
%	\addtocontents{toc}{\protect\newpage}
	\part{Github Desktop}
	\noptcrule
	\minitoc				


% ----------------------------------------------------------------------------- 	Github Desktop 개요
%										
% -----------------------------------------------------------------------------										
	\section{Github Desktop 개요}


% ----------------------------------------------------------------------------- 	Github Desktop 설치
%										
% -----------------------------------------------------------------------------										
	\section{Github Desktop 설치}


% ----------------------------------------------------------------------------- 	Github Desktop 메인 화면
%										
% -----------------------------------------------------------------------------										
	\section{Github Desktop 메인 화면}


% ----------------------------------------------------------------------------- 	Github Desktop 원격저장소 생성 화면
%										
% -----------------------------------------------------------------------------										
	\section{Github Desktop 원격저장소 생성 화면}



% ----------------------------------------------------------------------------- 	Github Desktop 다양한 활용법
%										
% -----------------------------------------------------------------------------										
	\section{Github Desktop 다양한 활용법}



%	================================================================== Part		소스 트리
%	\addtocontents{toc}{\protect\newpage}
	\part{소스 트리}
	\noptcrule
	\minitoc				


% ----------------------------------------------------------------------------- 	소스 트리 개요		소스트리
%										
% -----------------------------------------------------------------------------										
	\section{소스 트리 개요}


% ----------------------------------------------------------------------------- 	소스 트리 설치
%										
% -----------------------------------------------------------------------------										
	\section{소스 트리 설치}

% ----------------------------------------------------------------------------- 	소스 트리 에서 GitHub 로그인하기
	\section{소스 트리 에서 GitHub 로그인하기}

% ----------------------------------------------------------------------------- 	소스 트리 화면 구성  48p
	\section{소스 트리 화면 구성 }


	\paragraph 	{로컬 저장소 목록}
	\paragraph 	{원격 저장소 목록}
	\paragraph 	{원격 저장소 클론}
	\paragraph 	{로컬 저장소 추가}
	\paragraph 	{로컬 저장소 생성}





%	================================================================== Part			용어해설
%	\addtocontents{toc}{\protect\newpage}
	\part{ 용어 해설  }
	\noptcrule
	\parttoc				


% ----------------------------------------------------------------------------- 	용어해설
%										
% -----------------------------------------------------------------------------										
	\section 	{용어 해설}


% ----------------------------------------------------------------------------- 	아틀라시안
%										
% -----------------------------------------------------------------------------										
	\section 	{아틀라시안}



% ----------------------------------------------------------------------------- 	비트 버킷
%										
% -----------------------------------------------------------------------------										
	\section 	{비트 버킷}


% ----------------------------------------------------------------------------- 	저장소
%										
% -----------------------------------------------------------------------------										
	\section 	{저장소 Repository}


% ----------------------------------------------------------------------------- 	Staging Area
%										
% -----------------------------------------------------------------------------										
	\section 	{Staging Area}



% ----------------------------------------------------------------------------- 	소프트웨어 개발방식
%										
% -----------------------------------------------------------------------------										
	\section 	{소프트웨어 개발방식}



% ----------------------------------------------------------------------------- 	DVGS 분산 버전 관리 시스템
%										
% -----------------------------------------------------------------------------										
	\section 	{DVGS 분산 버전 관리 시스템}


% ----------------------------------------------------------------------------- 	스냅샷
%										
% -----------------------------------------------------------------------------										
	\section 	{스냅샷}




%	================================================================== Part		참고문헌
%	\addtocontents{toc}{\protect\newpage}
	\part{ 참고문헌 }
	\noptcrule
	\parttoc				


% ----------------------------------------------------------------------------- 	소프트웨어 개발에 GitHub 활용하기
%										
% -----------------------------------------------------------------------------										
	\section 	{소프트웨어 개발에 GitHub 활용하기}




% ----------------------------------------------------------------------------- 	지옥에서 온 문서관리자
%										
% -----------------------------------------------------------------------------										
	\section 	{지옥에서 온 문서관리자}






% ----------------------------------------------------------------------------- 	팀 개발을 위한 	Git Github 시작하기 문서관리자	팀개발
%										
% -----------------------------------------------------------------------------										
	\chapter 	{팀 개발을 위한 	Git Github 시작하기 문서관리자}
	\minitoc

		\section 	{머리말}
		\section 	{동영상 안내}

		\begin{itemize}
			\item 	http://bit.ly/do-it-git1
			\item 	http://bit.ly/do-it-git2
			\item 	http://bit.ly/do-it-git3
			\item 	http://bit.ly/do-it-git4
			\item 	http://bit.ly/do-it-git5
		\end{itemize}



	 	\section{	지옥에서 온 문서관리자	}		
						
		\section{	01장 깃 시작하기	}		
		\subsection{	01-1 지옥에서 온 관리자, 깃	}		
		\subsection{	01-2 깃 설치하기	}		
		\subsection{	01-3 리눅스 명령 연습하기	}		
		\subsection{	01장에서 꼭 기억해야 할 명령	}		
						
		\section{	02장 깃으로 버전 관리하기	}		
		\subsection{	02-1 깃 저장소 만들기	}		
		\subsection{	02-2 버전 만들기	}		
		\subsection{	02-3 커밋 내용 확인하기	}		
		\subsection{	02-4 버전 만드는 단계마다 파일 상태 알아보기	}		
		\subsection{	02-5 작업 되돌리기	}		
		\subsection{	02장에서 꼭 기억해야 할 명령	}		
						
		\section{	03장 깃과 브랜치	}		
		\subsection{	03-1 브랜치란?	}		
		\subsection{	03-2 브랜치 만들기	}		
		\subsection{	03-3 브랜치 정보 확인하기	}		
		\subsection{	03-4 브랜치 병합하기	}		
		\subsection{	03-5 브랜치 관리하기	}		
		\subsection{	03장에서 꼭 기억해야 할 명령	}		
						
		\section{	04장 깃허브로 백업하기	}		
		\subsection{	04-1 원격 저장소와 깃허브	}		
		\subsection{	04-2 깃허브 시작하기	}		
		\subsection{	04-3 지역 저장소를 원격 저장소에 연결하기	}		
		\subsection{	04-4 원격 저장소에 올리기 및 내려받기	}		
		\subsection{	04-5 깃허브에 SSH 원격 접속하기	}		
		\subsection{	04장에서 꼭 기억해야 할 명령	}		
						
		\section{	05장 깃허브로 협업하기	}		
		\subsection{	05-1 여러 컴퓨터에서 깃허브 저장소 함께 사용하기	}		
		\subsection{	05-2 원격 브랜치 정보 가져오기	}		
		\subsection{	05-3 협업의 기본 알아보기	}		
		\subsection{	05-4 협업에서 브랜치 사용하기	}		
		\subsection{	05장에서 꼭 기억해야 할 명령	}		
						
		\section{	06장 깃허브에서 개발자와 소통하기	}		
		\subsection{	06-1 깃허브 프로필 관리하기	}		
		\subsection{	06-2 README 파일 작성하기	}		
		\subsection{	06-3 오픈 소스 프로젝트에 기여하기	}		
		\subsection{	06-4 깃허브에 개인 블로그 만들기	}		
						
		\section{	실무 밀착 꿀팁!	}		




% ----------------------------------------------------------------------------- 	만들면서 배우는 Git Github 입문
%										
% -----------------------------------------------------------------------------										
	\chapter 	{만들면서 배우는 Git Github 입문}

	\section 	{만들면서 배우는 Git Github 입문}




% ----------------------------------------------------------------------------- 	분산 버전 관리 Git 사용설명서
%										
% -----------------------------------------------------------------------------										
	\section 	{분산 버전 관리 Git 사용설명서}








%	================================================================== 		깃 실습
	\addtocontents{toc}{\protect\newpage}
	\chapter {깃 실습}
	\noptcrule

%	\newpage	
	\minitoc
%	\secttoc

% ----------------------------------------------------------------------------- 	 Git 실습
%	
% -----------------------------------------------------------------------------	
	\section 	{ Git 실습}

\paragraph{기본설정}
		.\\


		% ------------------------------------------------- tcolorbox package
		\begin{tcolorbox}		[
%								colback=green!5,
								colback=red!5!white,
								colframe=red!75!black,
%								colframe=green!40!black,
								title=깃 기본 설정
								]
			\$git config --glabal user.name  "kim dae hee 5609 " \\
			\$git config --glabal user.email  "h 010 3839 5609 @ gmail . com" \\
			\$git config --glabal color.ui true
		\end{tcolorbox}


\paragraph{실습 디렉토리 준비}
		.\\

		\begin{tcolorbox}
			mkdir hello \\
			cd hello 
		\end{tcolorbox}


\paragraph{Git 저장소 초기화}
		.\\


		% ------------------------------------------------- tcolorbox package
			\begin{tcolorbox}		[
									title=깃 저장소 초기화
									]
									git init
			\end{tcolorbox}

		\begin{tcolorbox}
			Initialized empty Git repository in C:/Users/김대희/Desktop/git/hello/.git/
		\end{tcolorbox}



\paragraph{Git 저장소 초기화 상태 확인}
		.\\

		% ------------------------------------------------- tcolorbox package
			\begin{tcolorbox}		[
									title=깃 저장소 초기화 상태 확인
									]
									git status
			\end{tcolorbox}
			현재 작업영역 상태 확인

		\begin{tcolorbox}
			On branch master \\
			No commit yet \\
			noting to commit (create/copy files and use "git add" to track)
		\end{tcolorbox}


\paragraph{새 파일 추가}
		.\\

		% ------------------------------------------------- tcolorbox package
			\begin{tcolorbox}		[
									title=새 파일 추가
									]
									vim hello.html
			\end{tcolorbox}


\paragraph{다시 상태 확인}
		.\\

		% ------------------------------------------------- tcolorbox package
			\begin{tcolorbox}		[
									title=다시 상태 확인
									]
									git status
			\end{tcolorbox}

		\begin{tcolorbox}
			On branch master \\
			No commit yet \\
			unTracked files :\\
			(use "git add <file>..." to include in what will be committed)\\
				hello.html
			noting added to commit but UnTracked files present ( use "git add" to track)
		\end{tcolorbox}


\paragraph{스테이지 영역에 추가}
		.\\

		% ------------------------------------------------- tcolorbox package
			\begin{tcolorbox}		[
									title=스테이지 영역에 추가
									]
									git add hello.html \\
									git status
			\end{tcolorbox}


		\begin{tcolorbox}
			On branch master \\
			No commit yet \\
			Changes to be commited :\\
			(use "git rm --cached <file>..."to UnStage)
			new file : hello.html
		\end{tcolorbox}

\paragraph{첫 커밋!}
		.\\

		% ------------------------------------------------- tcolorbox package
			\begin{tcolorbox}		[
									title=첫 커밋!
									]
									git commit -m "add hello.html"
			\end{tcolorbox}
			커밋 하면서 설명을 남긴다

		\begin{tcolorbox}
			( master (root-commit) 87bad9b ) add hello.html\\
			1 file changed, 16 insertions(+) \\
			creat mode 100644 hello.html
		\end{tcolorbox}
		

\paragraph{첫 커밋 후 상태 확인}
		.\\

		% ------------------------------------------------- tcolorbox package
			\begin{tcolorbox}		[
									title=첫 커밋 후 상태 확인
									]
									git ststus
			\end{tcolorbox}

		\begin{tcolorbox}
			On branch master \\
			nothing to commit, working tree clean
		\end{tcolorbox}


\paragraph{커밋 로그 확인}
		.\\

		% ------------------------------------------------- tcolorbox package
			\begin{tcolorbox}		[
									title=커밋 로그 확인
									]
									git log
			\end{tcolorbox}

		\begin{tcolorbox}
			commit (HEAD -> master) \\
			Author : 김대희 < >\\
			Date : 			\\
			add hello.html
		\end{tcolorbox}




%	================================================================== 		팀 개발의 위한 Git GitHub 시작하기
%
%	일단 여기에 편집하고 나중에 위치 이동 
%	맨 마지막이라 읻오의 편리 때문에 이곳에서 편집

	\addtocontents{toc}{\protect\newpage}
	\chapter {팀 개발의 위한 Git GitHub 시작하기}
	\noptcrule

%	\newpage	
	\minitoc
%	\secttoc

% ----------------------------------------------------------------------------- 	 빠른 실습으로 감 익히기
%	
% -----------------------------------------------------------------------------	
	\section 	{빠른 실습으로 감익히기}

% ----------------------------------------------------------------------------- 	 Git 그리고 GitHub
%	
% -----------------------------------------------------------------------------	01
	\section 	{01 Git 그리고 GitHub}
		
		\subsection 	{버전관리란 무엇인가?}
		\subsection 	{Git 그리고 GitHub}
		\subsection 	{GitHub 가입하기 }

% -----------------------------------------------------------------------------	02
	\section 	{02 Git을 설치하고 로컬 저장소에서 커밋 관리하기}

		\subsection 	{내 컴퓨터에 Git 설치하기}

			\paragraph{다운로드} 
			.\\
			\begin{tcolorbox}
				https://git-scm.com/downloads
			\end{tcolorbox}

			\paragraph{설치 확인}
			.\\
			\begin{tcolorbox}
				Git Bash
			\end{tcolorbox}


		\subsection 	{로컬 저장소 만들기}

			\paragraph{폴더 작성} 
			.\\

			\begin{tcolorbox}
				iTshirt-cat
			\end{tcolorbox}

			\paragraph{readme.txt} 
			.\\

			\begin{tcolorbox}
				개발자 티셔츠 쇼핑몰 오픈소스 
			\end{tcolorbox}

			\paragraph 	{폴더 초기화}
			.\\

			\begin{tcolorbox}
				git init
			\end{tcolorbox}


			\begin{tcolorbox}
				git config --global user.email 	"h01038395609@gmail.com" \\
				git config --global user.name	"kimdaehee5609"
			\end{tcolorbox}


			\begin{tcolorbox}
				git config --list \\
				git config --global -e \\
				git config --global core.autocrlf
			\end{tcolorbox}


		\subsection 	{첫번째 커밋 만들기}

			\begin{tcolorbox}
				git add README.txt
			\end{tcolorbox}



			\begin{tcolorbox}
				git commit -m "사이트 설명 추가"
			\end{tcolorbox}

			\paragraph{readme.txt} 
			.\\


		\subsection 	{두번째 커밋 만들기}

			\paragraph{readme.txt 내용 수정} 
			.\\

			\begin{tcolorbox}
				개발자 티셔츠 쇼핑몰 오픈소스 짱
			\end{tcolorbox}

			\begin{tcolorbox}
				git add  README.txt
			\end{tcolorbox}

			\begin{tcolorbox}
				git commit -m "설명 업데이터"
			\end{tcolorbox}


		\subsection 	{다른 커밋으로 시간 여행하기}

			\begin{tcolorbox}
				git log
			\end{tcolorbox}

			커밋의 아이디 확인 (7자리)

			\paragraph{readme.txt 내용 되돌리기} 
			.\\

			\begin{tcolorbox}
				git check out 첫번째 커밋 
			\end{tcolorbox}
			readme.txt의 수정사항"짱"이 없어짐

			\begin{tcolorbox}
				git check out -
			\end{tcolorbox}
			readme.txt의 수정사항"짱"이 나타남


% -----------------------------------------------------------------------------	03
	\section 	{03 GitHub 원격저장소에 커밋 올리기 21p}

		\subsection 	{원격 저장소 만들기}


			\paragraph{GitHub에 접속} 
			.\\

			\begin{tcolorbox}
				github.com 접속 \\
				github.com 로그인 
			\end{tcolorbox}

			\paragraph{저장소 생성}
			.\\

			\begin{tcolorbox}
				우측 상단의 + 아이콘 선택 \\
				new Repository 메뉴 선택 \\
				Repository name : iTshirt \\
				description : it인을 위한 티셔츠 쇼핑몰 오픈소스
			\end{tcolorbox}


			\paragraph{저장소 삭제}
			.\\

			\begin{tcolorbox}
				우측 상단의 setting \\
				Dange Zone \\
				Delete this repository  \\
				비밀번호 : h . 789 456 123  \\
			\end{tcolorbox}


		\subsection 	{원격 저장소에 커밋 올리기}

			\paragraph{원격저장소 주소 알아내기}
			.\\

			\begin{tcolorbox}
				Quick Setup  맨 오른쪽 버튼을 클릭하면 주소가 복사됨 
			\end{tcolorbox}


			\paragraph{원격저장소 주소 입력}
			.\\
			\begin{tcolorbox}
				git remote add origin https://github.com/cat-hanbit/itshirt.git \\
									https://github.com/kimdaehee5609/iTshirt.git
			\end{tcolorbox}

			\paragraph{원격저장소에 올리기}
			.\\
			\begin{tcolorbox}
				git push origin master
			\end{tcolorbox}



% -----------------------------------------------------------------------------	04 작업작업
	\section 	{04 GitHub 원격저장소의 커밋을 로컬저장소에 내려받기}

		\subsection 	{원격저장소의 커밋을 로컬저장소에 내려받기}

			\paragraph{저장 폴더 만들기}
			.\\

			\begin{tcolorbox}
				iTshirt-oct
			\end{tcolorbox}

			\begin{tcolorbox}
				Git Bash Here
			\end{tcolorbox}

			\paragraph{원격 저장소의 주소 얻어오기}
			.\\
			Code 원격 저장소 주소를 오른쪽에 있는 버튼을  클릭해서 원격저장소 주소를 복사한다.\\
			https://github.com/kimdaehee5609/iTshirt.git


			\paragraph{colon}
			.\\
			\begin{tcolorbox}
				git clone	 	http://github.com/cat-hanbit/itshirt.git .\\
							https://github.com/kimdaehee5609/iTshirt.git
			\end{tcolorbox}
			한 칸 뛰우고 마침표


			\paragraph{readme.txt 내용 확인하기} 
			.\\

			\paragraph{readme.txt 내용 수정} 
			.\\

			\begin{tcolorbox}
				개발자 목록\\
				1. 고양이 \\
				2. 문어
			\end{tcolorbox}

			\begin{tcolorbox}
				git bash
			\end{tcolorbox}

			\begin{tcolorbox}
				git add README.txt
			\end{tcolorbox}

			\begin{tcolorbox}
				git commit -m "개발자 목록 추가"
			\end{tcolorbox}

			\begin{tcolorbox}
				git push origin master
			\end{tcolorbox}

		
		\subsection 	{원격저장소의 새로운 커밋을 로컬저장소에 갱신하기}

			\begin{tcolorbox}
				git pull origin master
			\end{tcolorbox}


		\subsection 	{잠시만요 복습좀하고 가죠}

		\paragraph{git} 깃 이라 읽고, 버전 관리 시스템입니다
		\paragraph{gitHub} 깃허브라고 일고, Git으로 관리하는 프로젝트를 올려둘수 있는 사이트 입니다.

		\paragraph{GUI} 그래픽유저 인테피이스, 즉 마우스를 클릭하여 사용하는 방식입니다.
		\paragraph{CLI} 커맨드 라인 인터페이스, 즉 명령어를 하나씩 입력하는 방식입니다

		\paragraph{git bash} CLI 방식으로 Git을 사용할 수 있는 환경입니다.
		\paragraph{커밋} 버전 괸리르통해 생성된 파일, 혹은 그 행위를 의미합니다.
		\paragraph{log 명령어} 지금 까지 만든 커밋을 모두 확인 합니다.
		\paragraph{체크아웃한다} checkout으로 원하는 지점으로 되돌릴수 있습니다.
												타임머신과 같다.
		\paragraph{로컬 저장소} Git 으로 버전 관리하는 내 컴퓨터 안의 폴더를 의미합니다
		\paragraph{원격 저장소} GitHub에서 협업하는 공간을 의미합니다
		\paragraph{레포지토리} 원격저장소를 의미합니다 
		\paragraph{푸시} 로컬저장소의 커밋을 원격저장소에 올리는 것
		\paragraph{풀} 원격저장소의 커밋을 로컬저장소로 내려받는 것









% ------------------------------------------------------------------------------
% End document
% ------------------------------------------------------------------------------
\end{document}


	\href{https://www.youtube.com/watch?v=SpqKCQZQBcc}{태양경배자세A}
	\href{https://www.youtube.com/watch?v=CL3czAIUDFY}{태양경배자세A}


https://docs.google.com/spreadsheets/d/1-wRuFU1OReWrtxkhaw9uh5mxouNYRP8YFgykMh2G_8c/edit#gid=0
+

https://seoyeongcokr-my.sharepoint.com/:f:/g/personal/02017_seoyoungeng_com/Ev8nnOI89D1LnYu90SGaVj0BTuckQ46vQe1HiVv-R4qeqQ?e=S3iAHi


		\begin{tcolorbox}[title=My title,
		colback=red!5!white,
		colframe=red!75!black,
		colbacktitle=yellow!50!red,
		coltitle=red!25!black,
		fonttitle=\bfseries,
		subtitle style={boxrule=0.4pt,
		colback=yellow!50!red!25!white} ]
		This is a \textbf{tcolorbox}.
		\tcbsubtitle{My subtitle}
		Further text.
		\tcbsubtitle{Second subtitle}
		Further text.
		\end{tcolorbox}
		

		\begin{tcbraster}[raster columns=3, raster equal height,
		size=small,colframe=red!50!black,colback=red!10!white,colbacktitle=red!50!white,
		title={Box \# \thetcbrasternum}]
			\begin{tcolorbox}First box\end{tcolorbox}
			\begin{tcolorbox}Second box\end{tcolorbox}
			\begin{tcolorbox}This is a box\\with a second line\end{tcolorbox}
			\begin{tcolorbox}Another box\end{tcolorbox}
			\begin{tcolorbox}A box again\end{tcolorbox}
		\end{tcbraster}